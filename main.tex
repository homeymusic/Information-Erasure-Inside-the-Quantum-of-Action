\documentclass[%
 aps, reprint, prl, amsmath,amssymb
]{revtex4-2}

\usepackage{graphicx}% Include figure files
\graphicspath{{figures/}} 
\usepackage{dcolumn}% Align table columns on decimal point
\usepackage{bm}% bold math
\usepackage{orcidlink}
\usepackage{mathtools}

% -------------- Watermark ----------------
\usepackage[style=iso]{datetime2}
\usepackage{FiraMono} 
\usepackage{tikz}
\usepackage{transparent} 

% Use shipout/background to ensure it stays behind text/figures
\AddToHook{shipout/background}{
  \begin{tikzpicture}[remember picture, overlay]
    \node [
      rotate=55,
      scale=1.2,
      text opacity=0.15,      
      color=gray!50,          
      font=\fontfamily{FiraMono-TLF}\selectfont\bfseries,
      align=center
    ] at (current page.center) {
        {\fontsize{60}{70}\selectfont WORKING DRAFT} \\ [0.8cm]
        {\Huge \DTMnow} \\
        {\Huge doi.org/10.17605/OSF.IO/EV8H6} \\ [0.8cm]
        {\Large Computational Complex Systems Laboratory} \\
        {\Large Homey Music, Detroit, USA} \\ [0.2cm]
         {\transparent{0.25}\includegraphics[width=1.0cm]{logo.png}}
    };
  \end{tikzpicture}
}
% -----------------------------------------


\DeclareMathOperator*{\argmin}{argmin}
\begin{document}
\title{Information Erasure Inside the Quantum of Action}

\begin{abstract}
By maximizing entropy through Landauer information erasure inside de Gosson's quantum of action—where states are operationally indistinguishable—we show that the continuum collapses into discrete conjugate observables. Consistent with the experimental difficulty in isolating pure states, numerical results confirm that as action increases, Schrödinger-eigenstates become vanishingly sparse configurations—not privileged solutions—among the expanding possibilities in phase space. At the entropic limit of action, the proposed model provides a thermodynamic-geometric correspondence to standard quantum mechanics.
\end{abstract}

\author{Brian S. Mulloy\orcidlink{0000-0002-1803-3172}}
\email{brian@homeymusic.com}
\affiliation{Computational Complex Systems Laboratory\\Homey Music, Corktown, Detroit, MI, USA}
\date{\today}

\maketitle

\section{Equations}

\subsection{The Information-Action Limit}

We show that the energy inside the quantum of action available for information erasure is limited by
\begin{equation}
\label{eq:InfoActionLimit}
b E \tau \le \pi \hbar ,
\end{equation}
where $b = |s^*|$ is the length of the binary encoding of the physical state, $E = k_B T \ln 2$ is the Landauer erasure energy, and $\tau$ is the characteristic time required to flip a single binary value to its orthogonal state. Here $\pi \hbar = \frac{\pi}{4} \Delta q \Delta p$ is the quantum of action, the minimal elliptical phase-space area of finite support; unlike statistical variance $\sigma_q \sigma_p$, these dimensions $\Delta q$ and $\Delta p$ define the hard Heisenberg boundaries where states are operationally indistinguishable.

\paragraph{Referencing Example:}
The information-action limit is captured by the inequality in Eq.~(\ref{eq:InfoActionLimit}).

\subsection{Information Erasure Inside the Quantum of Action}

Landauer erasure within a conjugate pair of action quanta is bounded by:
\begin{equation}
\label{eq:SymplecticBounds}
|q_? - \mathbf{Q}^*| \le \mathcal{P}^{-1}, 
\quad 
|p_? - \mathbf{P}^*| \le \mathcal{Q}^{-1},
\end{equation}
where the operationally indistinguishable microstates are represented as $\mu_q \coloneqq q_0 q_?$ and $\mu_p \coloneqq p_0 p_?$, while the macrostates $\mathbf{Q}^*$ and $\mathbf{P}^*$, supported by the energy available in each orthogonal quanta of action, are defined as $q \coloneqq q_0 \mathbf{Q}^*$ and $p \coloneqq p_0 \mathbf{P}^*$. The maximal extent and maximal momentum of the system are represented as $\Delta q \coloneqq q_0 \mathcal{Q}$ and $\Delta p \coloneqq p_0 \mathcal{P}$ 

\paragraph{Referencing Example:}
The symplectic squeezing is captured by Eq.~(\ref{eq:SymplecticBounds}).

\section{Foundational Manuscripts}

\begin{itemize}
    \item \textbf{Geometry}
    \begin{itemize}
        \item Heisenberg \cite{Heisenberg1927}
        \item de Gosson \cite{DeGosson2003}
    \end{itemize}
    
    \item \textbf{Thermodynamics}
    \begin{itemize}
        \item Landauer \cite{Landauer1961}
        \item Plenio and Vitelli \cite{PlenioVitelli2001}
        \item Margolus and Levitin \cite{MargolusLevitin1998}
    \end{itemize}
    
    \item \textbf{Methodology}
    \begin{itemize}
        \item Stern \cite{Stern1858}
        \item Brocot \cite{Brocot1861}
        \item Graham, Knuth, and Patashnik \cite{ConcreteMath}
        \item Stolzenburg \cite{Stolzenburg2015}
        \item Aiylam \cite{Aiylam2013}
    \end{itemize}
\end{itemize}

\bibliography{main} 
\end{document}
