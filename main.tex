\documentclass[%
 aps, reprint, prl, amsmath,amssymb
]{revtex4-2}

\usepackage{graphicx}% Include figure files
\usepackage{dcolumn}% Align table columns on decimal point
\usepackage{bm}% bold math
\usepackage{orcidlink}
\usepackage{svg}
\usepackage{mathtools}
\usepackage{amsmath}
\DeclareMathOperator*{\argmin}{argmin}
\begin{document}

\title{Phase-Space Quantization from Information Erasure}

\begin{abstract}
When the principle of maximum algorithmic entropy is applied to the quantum of action ($h/2$) in symplectic phase space, the Second Law of Thermodynamics  eliminates physically surplus information. We demonstrate that this physical imperative forces a continuous phase space to collapse into a discrete geometry. Numerical analysis of the harmonic oscillator confirms this geometry recovers the marginals of the Wigner quasiprobability distribution. These results suggest phase-space quantization is a thermodynamic necessity, providing to quantum mechanics an algorithmic correspondence that operates without the postulates of probability waves or superpositions.
\end{abstract}

\author{Brian S. Mulloy\orcidlink{0000-0002-1803-3172}}
\email{brian@mulloy.us}
\affiliation{Corktown, Detroit, MI, USA}
\date{\today}

\maketitle
\section{Rational Topology}

The foundation of our analysis relies on the Algorithmic Free Energy $\mathbf{F}(X, P)$, which determines the system's actual capacity to do work from an individual physical state, $(X, P)$. Based on the framework of Algorithmic Thermodynamics [1], we define this capacity via the Helmholtz Algorithmic Free Energy (analogous to Eq. 56), where the state's complexity is given by the length of the SB path $\gamma$ (e.g., $\mathbf{x_{\gamma}} = \text{"LRLRRRLLR"}$). The algorithmic free energy, in units of energy, is:

$$\mathbf{F}(X, P) := E(X, P) - K(\mathbf{x_{\gamma}}|\tilde{P}) \cdot k_B T \ln 2 \quad \text{(A)}$$

Here, $E(X, P)$ is the internal energy of the QHO state, and $K(\mathbf{x_{\gamma}}|\tilde{P})$ is the Conditional Algorithmic Entropy of the path $\mathbf{x_{\gamma}}$ (in bits), conditioned on the dynamics $\tilde{P}$ of the tree search. The term $k_B T \ln 2$ represents the Landauer energy (Eq. 3). Halting occurs when the cost of irreversible information gain (increase in $K(\mathbf{x_{\gamma}}|\tilde{P})$) cannot be met by the available $\Delta F$, consistent with the Algorithmic Second Law (Corollary 4, Eq. 58).


In the quantum blob framework, the Stern-Brocot tree provides a unique isomorphism where the path length $K(x)$ serves as the fundamental measure of structural complexity. Unlike Farey sequences, this encoding identifies $x$ as a specific sequence of binary decisions $\{\sigma_i\} \in \{L, R\}^K$, mapping the discrete rational lattice onto the real line through the relation:
\begin{equation}
    x \cong \lim_{K(x) \to \infty} \text{SB}(\sigma_1, \sigma_2, \dots, \sigma_K)
\end{equation}
The motivating equation $S_{\pi}(x) := K(x) + \log \pi(x)$ thus represents a total entropy where the path length $K(x)$ is the primary coordinate of existence, while the density $\pi(x)$ accounts for the local branching measure within the isomorphic space:
\begin{equation}
    K(x) = \text{depth}_{\text{SB}}(x) \implies S_{\pi}(x) \text{ is invariant under binary path transformations.}
\end{equation}


The system complexity is governed by the state function $S_{\pi}(x) := K(x) + \log \pi(x)$, where $K(x)$ represents the structural depth within the Stern-Brocot tree. For a rational $x = [a_0; a_1, \dots, a_n]$, this complexity is the sum of its continued fraction coefficients:
\begin{equation}
    K(x) = \left( \sum_{i=0}^{n} a_i \right) - 1.
\end{equation}
For general $x = p/q$, the path length $K(x)$ represents the number of binary mediant search steps, scaling logarithmically with the denominator as:
\begin{equation}
    K(p/q) \sim \mathcal{O}(\log q).
\end{equation}


Following Ebtekar and Hutter (2025),

\begin{equation}
    S_{\pi}(x) := K(x) + \log \pi(x)
\end{equation}
.


The Gács algorithmic entropy requires a prefix-free encoding to satisfy the fluctuation theorems of algorithmic thermodynamics (Ebtekar and Hutter, 2025). We therefore define the total information cost as the sum of the Stern-Brocot path length and its prefix-free header:
$$H_{\text{Gács}}(Q_0) := |B| + 2\log_2 |B| + 1$$
In this formulation, $$|B|$$ represents the configurational entropy, while the logarithmic terms account for the algorithmic cost of state specification.


Following the framework of algorithmic thermodynamics, we identify the algorithmic entropy $$H_{\text{algo}}$$ of a rational state $$q = a/b$$ as the length of its prefix-free Stern-Brocot encoding $$B$$:
$$H_{\text{algo}}(q) := |B|$$
In this discrete representation, the entropy is strictly equivalent to the Kolmogorov complexity of the state, $$H_{\text{algo}}(q) = K(q)$$. Within the phase-space interval enforced by the squeezing factor $$\xi_p$$, the principle of minimum algorithmic entropy selects the unique rational state:
$$Q_0 = \text{argmin} \{ H_{\text{algo}}(q) : \left| \frac{x}{x_0} - q \right| < \frac{1}{\xi_p} \}$$

% --- Physical Definition of Momentum Squeezing ---
We define the momentum uncertainty through the dimensionless squeezing scaling factor $$\xi_p$$ as:
$$\Delta p := \xi_p p_0$$
where $$p_0 = \hbar / (2x_0)$$ represents the vacuum noise level. 

% --- The Topological Phase-Space Constraint ---
Following the symplectic geometry of de Gosson's quantum blobs, the positional resolution relative to a fundamental rational $$Q_0=a/b \in \mathbb{Q}$$ is governed by the beautiful relationship:
$$\left| \frac{x}{x_0} - \frac{a}{b} \right| < \frac{1}{\xi_p}$$

Within the interval defined by the squeezing scaling $$\xi_p$$, there exist infinitely many rationals $$q \in \mathbb{Q}$$. Following Ebtekar and Hutter (2024), we define the algorithmic entropy of a state as its Kolmogorov complexity $$H_{\text{algo}}(q) \approx K(q)$$. We identify the unique rational $$Q_0 = a/b$$ that minimizes this entropy:
$$Q_0 = \text{unique } \min_{q \in \mathbb{Q}} \{ H_{\text{algo}}(q) : \left| \frac{x}{x_0} - q \right| < \frac{1}{\xi_p} \}$$
The binary path length $$|B|$$ required to reach a node in the Stern-Brocot tree defines the prefix-free encoding of the rational state. We establish the physical significance of this length by identifying it with the algorithmic entropy $$H_{\text{algo}}$$ as defined by Ebtekar and Hutter (2025):
$$H_{\text{algo}}(a/b) \simeq |B|$$
Within the uncertainty region defined by the squeezing $$\xi_p$$, the system must occupy the state that minimizes this entropy. Because the path length $$|B|$$ is strictly monotonic with respect to mediant depth, there exists a unique rational $$Q_0$$ that satisfies:
$$H_{\text{algo}}(Q_0) = \min_{q \in \mathbb{Q}} \{ H_{\text{algo}}(q) : \left| \frac{x}{x_0} - q \right| < \frac{1}{\xi_p} \}$$
This minimization identifies $$Q_0$$ as the thermodynamic ground state of the rational topology, providing a rigorous information-theoretic origin for the quantum harmonic oscillator.

Because the Stern-Brocot path length $$|B|$$ is strictly increasing with the mediant depth, the rational with the shortest path is the unique ancestor in the SB tree that first enters the localized pixel.

% --- Derivation from Algorithmic Entropy ---
By applying the principles of algorithmic thermodynamics to the Stern-Brocot tree, we minimize the algorithmic entropy of the phase-space bit-string to find the explicit form of the scaling factor:
$$\xi_p = \frac{1}{2} \sqrt{n^2(n^2 + 2)}$$

% --- Final Scaling Result ---
In the limit of large quantum number $$n$$, the squeezing behaves as $$\xi_p \approx n^2 / 2$$. This confirms that the phase sensitivity $$\Delta \theta \propto 1/\xi_p$$ achieves a super-Heisenberg scaling of $$N^{-2}$$.


\section{The Stern-Brocot Triadic Isomorphism}
The Stern-Brocot tree $\mathcal{T}$ serves as a tripartite bridge between physical approximation, rational number theory, and algorithmic complexity. It provides a unique, one-to-one mapping between the set of rational numbers $q \in \mathbb{Q}$ and the set of finite binary strings $s \in \{L, R\}^*$, governed by the approximation error $\epsilon$ of a real-valued physical state $x \in \mathbb{R}$:

\begin{equation}
|x - q| < \epsilon \longleftrightarrow q = \frac{a}{b} \longleftrightarrow s_q
\end{equation}

In this framework, the tree operates in three simultaneous directions:
\begin{enumerate}
    \item \textbf{Real Approximation Error:} The physical distance $|x - q|$ provides the stopping criterion for the search, anchored to the Heisenberg limit $\Gamma_0$.
    \item \textbf{Rational Representation:} The fraction $q = a/b$ provides the discrete physical coordinate.
    \item \textbf{Binary Isomorphism:} The bit string $s_q$ provides the algorithmic path (the "instruction set") required to reach that coordinate.
\end{enumerate}

Following the work of Bates, Bunder, and Tognetti \cite{Bates2010}, this structure is the most physically efficient mechanism for rational estimation. Because the tree is strictly ordered and isomorphic to continued fraction expansions, it minimizes the computational distance ($s_q$) required to reach a specific physical resolution ($\epsilon$).

\section{The Foundational Equations}

\subsection{Heisenberg Uncertainty Principle}
The fundamental limit of phase-space resolution for a single degree of freedom is defined by the uncertainty product:
\begin{equation}
\Delta x \Delta p \ge \frac{\hbar}{2}
\end{equation}

\subsection{Minimal Heisenberg Cell}
We define the minimal physical domain $\Gamma_0$ as the phase-space region satisfying the equality of the uncertainty principle:
\begin{equation}
\Delta x \Delta p = \frac{\hbar}{2} \implies \Gamma_0
\end{equation}

\subsection{Landauer Principle}The physical link between information and energy is established by the minimum heat dissipation required to specify the algorithmic path \(s_{q}\). The energy cost scales with the length of the string:\begin{equation}E(s_q) \ge |s_q| \cdot k_B T \ln 2\end{equation}

\subsection{Gács Entropy}
The algorithmic entropy of a rational state $q$ is the complexity of its Stern-Brocot path $s_q$, representing the minimal information required to specify that state:
\begin{equation}
H(q) = K(s_q) + O(1)
\end{equation}

\subsection{Fundamental Fraction}
The Fundamental Fraction $Q_0$ is the unique rational point within the Heisenberg cell that minimizes the algorithmic entropy:
\begin{equation}
Q_0 = \arg\min_{q \in \Gamma_0 \cap \mathbb{Q}} H(q)
\end{equation}

\section{Introduction}

The physical limits of phase space are governed by the Heisenberg uncertainty principle, $\Delta x \Delta p \ge \hbar/2$. Within the saturated, minimal Heisenberg phase-space cell $\Delta x \Delta p = \hbar / 2$, all states are operationally indistinguishable [Fig.~\ref{fig:LandauerHeisenbergCell}]. 

Landauer's principle establishes a fundamental lower bound on the energy required to process information: any logically irreversible manipulation of a bit, such as its erasure, must be accompanied by a heat dissipation of at least $E \ge k_B T \ln 2$v.

Landauer's principle establishes a fundamental lower bound on the energy required to process information: any logically irreversible manipulation of a bit, such as its erasure, must be accompanied by a heat dissipation of at least $E \ge k_B T \ln 2$.

These two principles—Heisenberg's uncertainty and Landauer's bound—converge at a profound intersection. Within the minimal Heisenberg cell, the universe faces an informational dilemma: any attempt to specify a state with arbitrary precision requires encoding increasingly complex rational approximations, each demanding additional bits of description. Yet Landauer's principle imposes a thermodynamic tax on this precision: each bit costs energy to process, maintain, or erase.

This naturally motivates a principle of thermodynamic parsimony. The second law of thermodynamics drives isolated systems toward maximum entropy—the macrostate with the most microstates. Equivalently, in the language of algorithmic information theory, physical systems prefer minimal description length (MDL) representations: nature "chooses" the encoding that minimizes algorithmic complexity (Rissanen, 1978; Grünwald, 2007). This preference emerges not from aesthetic simplicity, but from thermodynamic necessity. High-precision states carry surplus information—bits that contribute negligibly to distinguishability within the Heisenberg cell but impose real energetic costs. The Landauer energy $E_0 = L(Q_0) k_B T \ln 2$ penalizes informational redundancy: states with longer descriptions are thermodynamically disfavored.

The minimum description length principle thus acts as nature's Occam's razor, enforced by the laws of thermodynamics. Just as equilibrium thermodynamics selects the maximum entropy distribution over phase space, the Landauer-Heisenberg constraint selects the minimum complexity representative within each quantum cell. The universe "prefers" MDL not by design, but because high-description-length states are exponentially suppressed by their thermodynamic cost.


\section{Back to the Flow}

We propose that states within the minimal Heisenberg cell are uniquely identified by a \textit{fundamental fraction} $Q_0 = a/b$. We identify $Q_0 \in \mathbb{Q}_0 \subset \mathbb{Q}$ as a canonically selected rational representative within the cell possessing the minimal description length. Formally, this corresponds to the minimal path length rational within the Stern--Brocot tree, which searches for rationals within the scaled approximation distance in phase space [Eq.~\eqref{eq:HeisenbergBox}].  Each branching step in the tree represents a binary choice, adding exactly one bit of information to the specification of the state. We define the description length $L(Q_{0})$ as the depth of the fraction within the tree; for a fraction $a/b$ with a bit-string representation of length $k$, $L(Q_{0})=k$. The Stern-Brocot tree generates the most balanced digital segments. This ensures that narrowing the search within a phase space cell follows the path of minimum description length while maintaining the most physically straight discretizations of the state's trajectory (Brlek et al., 2020, Berstel and de Luca, 1997, Andrea Frosini and Lama Tarsissi 2020).

\begin{figure}[tbp]
    \centering
    \includesvg[width=0.62\linewidth]{GacsLandauerHeisenbergCell}
    \caption{\textbf{The Gács--Landauer--Heisenberg Phase-Space Cell.} Within the minimal cell $\Delta x \Delta p = \hbar/2$, all states are operationally indistinguishable. We identify the physical state with its \textit{fundamental fraction}, the minimal description length rational representative $a/b \in \mathbb{Q}_0 \subset \mathbb{Q}$ (solid dot), obtained as the minimal path length rational within the Stern--Brocot tree. The dashed line illustrates the thermodynamic reduction of high-precision surplus information (faint point) to the minimal description length state $a/b$.}
    \label{fig:GacsLandauerHeisenbergCell}
\end{figure}

 By identifying the physical state with the minimal description length rational, the system discards the non-physical surplus information, thereby minimizing the thermodynamic cost. Consequently, this two-fold Gács--Landauer--Heisenberg constraint suggests that quantization is not merely an empirical observation, but a thermodynamic necessity: the quantum of action and the cost of information act in tandem to bound the resolution of phase space to a finite, discrete set of fundamental fractions.


\section*{The Gács--Landauer--Heisenberg Constraint}

We define $\mathcal{A}_0 \coloneqq \hbar/2$ as the fundamental unit of action, or the phase-space "quantum pixel." Within the operational limit $\Delta x \Delta p = \mathcal{A}_0$, we define the approximation error $\delta \tilde{x}$ relative to the radial action-scale resolution $r_{\mathcal{A}}$:
\begin{align}
    \delta \tilde{x} &\coloneqq x_0 \left| \frac{x}{x_0} - \frac{a}{b} \right|, \\
    \Delta p &\coloneqq r_{\mathcal{A}} \frac{\mathcal{A}_0}{x_0}.
\end{align}
The primary selection rule for the fundamental fraction $a/b \in \mathbb{Q}_0$ is then:
\begin{equation}
\label{eq:HeisenbergBox}
\left| \frac{x}{x_0} - \frac{a}{b} \right| < \frac{1}{r_{\mathcal{A}}} \implies \delta \tilde{x} \Delta p < \mathcal{A}_0.
\end{equation}
For the Quantum Harmonic Oscillator, $r_{\mathcal{A}}$ corresponds to the total number of action pixels required to resolve the state, such that $r_{\mathcal{A}} = 2n+1$. This establishes a direct mapping between the Stern--Brocot tree depth and the quantized energy spectrum.


\section{radial $r_{n_x,n_p}$}

The operational limit of the phase-space cell is defined by the Heisenberg equality $\Delta x \Delta p = \hbar/2$. Within this bound, we define the approximation error $\delta \tilde{x}$ and the momentum uncertainty $\Delta p$ relative to the system's characteristic scales:
\begin{align}
    \delta \tilde{x} &\coloneqq x_0 \left| \frac{x}{x_0} - \frac{a}{b} \right|, \\
    \Delta p &\coloneqq r_{np} p_0 = r_{np} \frac{\hbar}{2 x_0},
\end{align}
where $x, x_0 \in \mathbb{R}$ represent the position and characteristic length, and $p_0$ represents the characteristic momentum. In this formulation, $r_{np} \in \mathbb{R}$ serves as the dimensionless radial resolution of the phase-space cell.

For the state to remain operationally indistinguishable from the fundamental fraction, it must satisfy $\delta \tilde{x} < \Delta x$, which implies the uncertainty constraint $\delta \tilde{x} \Delta p < \hbar/2$. This leads to our primary selection rule for the fundamental fraction $a/b \in \mathbb{Q}_0$:
\begin{equation}
\label{eq:HeisenbergBox}
\left| \frac{x}{x_0} - \frac{a}{b} \right| < \frac{1}{r_{np}} \implies \delta \tilde{x} \Delta p < \frac{\hbar}{2}.
\end{equation}

Geometrically, this 1D bound represents the transversal of a phase-space orbit; the constraint $\delta \tilde{x} < 1/r_{np}$ ensures the rational point falls within the area of the characteristic action ellipse. For the symmetric Quantum Harmonic Oscillator, the parameter $r_{np}$ scales with the total quantized action such that $r_{np} \approx 2n + 1$. This implies that as the system occupies higher energy states, the informational resolution required to identify the fundamental fraction increases proportionally to the total phase-space area.

The informational cost is quantified by the length function $L(Q_0)$, defined as the number of bits (left or right moves) required to reach $Q_0$ from the root of the Stern--Brocot tree:
\begin{equation}
L(Q_0) = \text{depth}(Q_0) \implies E_0 = L(Q_0) k_B T \ln 2.
\end{equation}


\section{linear $n_{p_0}$}

The operational limit of the phase-space cell is defined by the Heisenberg equality $\Delta x \Delta p = \hbar/2$. Within this bound, we define the approximation error $\delta \tilde{x}$ and the momentum uncertainty $\Delta p$ relative to the system's characteristic scales:
\begin{align}
    \delta \tilde{x} &\coloneqq x_0 \left| \frac{x}{x_0} - \frac{a}{b} \right|, \\
    \Delta p &\coloneqq n_{p_0} p_0 = n_{p_0} \frac{\hbar}{2 x_0},
\end{align}
where $x, x_0 \in \mathbb{R}$ represent the position and characteristic length, $p_0$ the characteristic momentum, and $n_{p_0} \in \mathbb{R}$ a dimensionless momentum uncertainty factor. 

For the state to remain operationally indistinguishable from the fundamental fraction, it must satisfy $\delta \tilde{x} < \Delta x$, which implies the uncertainty constraint $\delta \tilde{x} \Delta p < \hbar/2$. This leads to our primary selection rule for the fundamental fraction $a/b \in \mathbb{Q}_0$:
\begin{equation}
\label{eq:HeisenbergBox}
\left| \frac{x}{x_0} - \frac{a}{b} \right| < \frac{1}{n_{p_0}} \implies \delta \tilde{x} \Delta p < \frac{\hbar}{2}.
\end{equation}

Geometrically, this 1D bound represents the radial extent of a phase-space orbit; the constraint $\delta \tilde{x} < 1/n_{p_0}$ ensures the rational point falls within the area of the characteristic action ellipse.


The informational cost is quantified by the length function $L(Q_0)$, defined as the number of bits (left or right moves) required to reach $Q_0$ from the root of the tree:
\begin{equation}
L(Q_0) = \text{depth}(Q_0) \implies E_0 = L(Q_0) k_B T \ln 2.
\end{equation}

\section*{Symplectic Justification of the 1D-2D Mapping}
The observed correspondence $n_{p_0} \approx 2n + 1$ arises from the symplectic symmetry of the QHO. In action-angle coordinates $(J, \theta)$, the phase-space area $A = 2\pi J$ is quantized as $A = (n + 1/2)h$. Because the oscillator trajectory is an invariant 1-torus, the 1D search constraint $\delta \tilde{x} < 1/n_{p_0}$ acts as a radial transversal of the phase-space disk. Under the dimensionless equipartition $\langle \tilde{x}^2 \rangle = \langle \tilde{p}^2 \rangle$, the linear resolution $n_{p_0}$ and the total action product $n_x n_p$ become numerically identical. Thus, the Stern-Brocot search at depth $L(Q_0)$ identifies the minimal rational address required to resolve the $n$-th action shell, effectively mapping Landauer informational cost to the quantized energy spectrum.



By satisfying Eq.~\eqref{eq:HeisenbergBox} using the minimal description length rational in the Stern--Brocot tree, the system identifies the fundamental fraction that minimizes the thermodynamic cost of information while remaining bounded by the quantum of action.

 To resolve an irrational value, or even a high-precision rational, an observer or environment would require infinite or at least prohibitive informational description length, demanding a corresponding prohibitive thermodynamic energy to erase or maintain that information.

Using $10^{8}$ computational data points, we will show that this fundamental fraction mapping—governed by the Landauer-Heisenberg constraint—recovers the emergent discrete structures of fundamental quantum systems. 

\textbf{[Forthcoming: Numerical verification of the canonical elementary quantum systems to be inserted here.]}

% connection to Bohr-Sommerfeld quantization rule
%
%
%
\section{Semiclassical Foundations: Bohr--Sommerfeld Mapping}
1. The "Formal Foundation" Approach
"We utilize the Phase-Space Action Topology, a framework made rigorous by the Bohr–Sommerfeld quantization rule, to define the physical boundaries within which our informational search operates."
2. The "Emergent Discreteness" Approach (Strongest for your novelty)
"By mapping our search onto the Phase-Space Action Topology, we demonstrate that the discrete energy levels formally predicted by Bohr–Sommerfeld emerge naturally from a requirement of minimal description length."
3. The "Methodological" Approach
"Our model adopts the Phase-Space Action Topology as its physical basis; while this domain is traditionally associated with Bohr–Sommerfeld semiclassical theory, we use it here to investigate the number-theoretic structure of the underlying Heisenberg cell."

To provide a first-principles derivation for the observed transitions, we align the dimensionless resolution factor $n_{p_0}$ with the semiclassical Action of the system. For a one-dimensional system, the Bohr--Sommerfeld quantization rule defines the allowed states through the phase-space area $S$:
\begin{equation}
\label{eq:BohrSommerfeld}
S = \oint p \, dq = \left( n + \frac{\mu}{4} \right) h,
\end{equation}
where $n \in \mathbb{N}_0$ is the quantum number and $\mu$ is the Maslov index. For the Quantum Harmonic Oscillator (QHO), the Maslov index $\mu = 2$ accounts for the phase shifts at the two classical turning points, yielding the quantized action $S = (n + 1/2)h$.

\subsection{The Action-Resolution Correspondence}
In the context of the informational Heisenberg box, we define the unit of action in terms of the fundamental quantum $\hbar/2$. Dividing the total QHO action by this fundamental unit reveals the \textbf{Occupancy Index}, $\mathcal{N}$:
\begin{equation}
\mathcal{N} = \frac{S}{\hbar/2} = \frac{(n + 1/2) \cdot 2\pi\hbar}{\hbar/2} = 2n + 1.
\end{equation}
Our experimental parameter $n_{p_0}$ maps directly to this occupancy index $\mathcal{N}$. Consequently, the sequence of observed phase transitions at $n_{p_0} \in \{1, 3, 5, \dots\}$ corresponds exactly to the discrete units of action required to resolve each successive quantum state $n$.

\subsection{Numerical Realization in the Stern--Brocot Tree}
The selection rule $\left| \frac{x}{x_0} - \frac{a}{b} \right| < \frac{1}{n_{p_0}}$ defines the maximum allowable approximation error $\delta \tilde{x}$ within the phase-space cell. As $n_{p_0}$ increases through the odd-integer sequence, the system must navigate deeper into the Stern--Brocot tree to identify a rational fraction $a/b$ that satisfies the tightening uncertainty bound. 

This process identifies the \textit{Minimal Description Length} (MDL) rational that "pierces the veil" of the minimal Heisenberg cell. The transitions in the distribution $H(\tilde{x}/x)$ emerge as the informational complexity $L(Q_0)$ jumps to accommodate the increased resolution required by the Bohr--Sommerfeld action of the next energy level. Thus, the integer nature of the quantum state emerges not from wave mechanics alone, but from the number-theoretic efficiency of rational approximation within the bounds of the quantum of action.

%%%%

% The \nocite command causes all entries in a bibliography to be printed out
% whether or not they are actually referenced in the text. This is appropriate
% for the sample file to show the different styles of references, but authors
% most likely will not want to use it.
\nocite{*}

\bibliography{apssamp}% Produces the bibliography via BibTeX.

\end{document}
