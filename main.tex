\documentclass[%
 aps, reprint, prl, amsmath,amssymb
]{revtex4-2}

\usepackage{graphicx}% Include figure files
\graphicspath{{figures/}} 
\usepackage{dcolumn}% Align table columns on decimal point
\usepackage{bm}% bold math
\usepackage{orcidlink}
\usepackage{mathtools}

% -------------- Watermark ----------------
\usepackage[style=iso]{datetime2}
\usepackage{FiraMono} 
\usepackage{tikz}
\usepackage{transparent} 

% Use shipout/background to ensure it stays behind text/figures
\AddToHook{shipout/background}{
  \begin{tikzpicture}[remember picture, overlay]
    \node [
      rotate=55,
      scale=1.2,
      text opacity=0.15,      
      color=gray!50,          
      font=\fontfamily{FiraMono-TLF}\selectfont\bfseries,
      align=center
    ] at (current page.center) {
        {\fontsize{60}{70}\selectfont WORKING DRAFT} \\ [0.8cm]
        {\Huge \DTMnow} \\
        {\Huge doi.org/10.17605/OSF.IO/EV8H6} \\ [0.8cm]
        {\Large Computational Complex Systems Laboratory} \\
        {\Large Homey Music, Detroit, USA} \\ [0.2cm]
         {\transparent{0.25}\includegraphics[width=1.0cm]{logo.png}}
    };
  \end{tikzpicture}
}
% -----------------------------------------


\DeclareMathOperator*{\argmin}{argmin}
\begin{document}
\title{Information Erasure Inside the Quantum of Action}

\begin{abstract}
By maximizing entropy through Landauer information erasure inside de Gosson's quantum of action—where states are operationally indistinguishable—we show that the continuum collapses into discrete conjugate observables. Consistent with the experimental difficulty in isolating pure states, numerical results confirm that as action increases, Schrödinger-eigenstates become vanishingly sparse configurations—not privileged solutions—among the expanding possibilities in phase space. At the entropic limit of action, the proposed model provides a thermodynamic-geometric correspondence to standard quantum mechanics.
\end{abstract}

\author{Brian S. Mulloy\orcidlink{0000-0002-1803-3172}}
\email{brian@homeymusic.com}
\affiliation{Computational Complex Systems Laboratory\\Homey Music, Corktown, Detroit, MI, USA}
\date{\today}

\maketitle

\section{Equations}

\subsection{The Information-Action Limit}

We propose a fundamental limit relating information energy to the quantum of action, expressed as
\begin{equation}
\label{eq:InfoActionLimit}
b E \tau \le \pi \hbar ,
\end{equation}
where $b = |s^*|$ is the length of the binary encoding of the phase-space coordinates, $E = k_B T \ln 2$ is the Landauer erasure energy, and $\tau$ is the characteristic time required to flip a single binary value to its orthogonal state. Here $\pi \hbar = \frac{\pi}{4} \Delta q \Delta p$ is the quantum of action, the minimal elliptical phase-space area of finite support; unlike statistical variance $\sigma_q \sigma_p$, these dimensions $\Delta q$ and $\Delta p$ define the hard Heisenberg boundaries where states are operationally indistinguishable.

\paragraph{Referencing Example:}
The information-action limit is captured by the inequality in Eq.~(\ref{eq:InfoActionLimit}).


\subsection{Information Erasure Inside the Quantum of Action}

Information erasure is the physical convergence of operationally indistinguishable phase-space micro states ($q_{?}, p_{?}$) toward the reset macro states ($\mathbf{Q}^*, \mathbf{P}^*$). This collapse is structurally constrained by the system's fundamental scales, where dimensionless fluctuations are bounded by the reciprocal of the maximal extent in the conjugate dimension:
\begin{equation}
\label{eq:SymplecticBounds}
|q_? - \mathbf{Q}^*| \le \mathcal{P}^{-1}, 
\quad 
|p_? - \mathbf{P}^*| \le \mathcal{Q}^{-1}
\end{equation}
Scale factors $q_0$ and $p_0$ define the physical position and momentum for both micro ($\mu_q, \mu_p$) and macro ($q, p$) states:
\begin{align}
\mu_q &\coloneqq q_0 q_? \quad \text{and} \quad \mu_p \coloneqq p_0 p_? \\
q &\coloneqq q_0 \mathbf{Q}^* \quad \text{and} \quad p \coloneqq p_0 \mathbf{P}^*
\end{align}
Here, $\mathcal{Q}$ and $\mathcal{P}$ represent the normalized maximal extent and maximal momentum of the system. These parameters define the \textbf{symplectic conjugate squeeze}: surplus bits in the encoding of $(q_?, p_?)$ not supported by the limited capacity of the Heisenberg cell $(\mathcal{P}^{-1}, \mathcal{Q}^{-1})$ are erased and collapse to $(\mathbf{Q}^*, \mathbf{P}^*)$. This reciprocal coupling ensures the elliptical area of the quantum of action remains invariant, preserving the fundamental action regardless of the specific Hamiltonian configuration.

\paragraph{Referencing Example:}
The symplectic squeezing is captured by Eq.~(\ref{eq:SymplecticBounds}).

\section{Foundational Manuscripts}

\begin{itemize}
    \item \textbf{Geometry}
    \begin{itemize}
        \item Heisenberg \cite{Heisenberg1927}
        \item de Gosson \cite{DeGosson2003}
    \end{itemize}
    
    \item \textbf{Thermodynamics}
    \begin{itemize}
        \item Landauer \cite{Landauer1961}
        \item Plenio and Vitelli \cite{PlenioVitelli2001}
        \item Margolus and Levitin \cite{MargolusLevitin1998}
    \end{itemize}
    
    \item \textbf{Methodology}
    \begin{itemize}
        \item Stern \cite{Stern1858}
        \item Brocot \cite{Brocot1861}
        \item Graham, Knuth, and Patashnik \cite{ConcreteMath}
        \item Stolzenburg \cite{Stolzenburg2015}
        \item Aiylam \cite{Aiylam2013}
    \end{itemize}
\end{itemize}

\bibliography{main} 
\end{document}
