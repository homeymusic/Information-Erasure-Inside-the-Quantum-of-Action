\documentclass[%
 aps, reprint, prl, amsmath,amssymb
]{revtex4-2}

\usepackage{graphicx}% Include figure files
\graphicspath{{figures/}} 
\usepackage{dcolumn}% Align table columns on decimal point
\usepackage{bm}% bold math
\usepackage{orcidlink}
\usepackage{mathtools}

% -------------- Watermark ----------------
\usepackage[style=iso]{datetime2}
\usepackage{FiraMono} 
\usepackage{tikz}

\AddToHook{shipout/foreground}{
  \begin{tikzpicture}[remember picture, overlay]
    \node [
      rotate=55,              % Restores the diagonal aspect-ratio alignment
      scale=1.2,              % Large impactful scale
      text opacity=0.18, 
      color=gray!80,
      font=\fontfamily{FiraMono-TLF}\selectfont\bfseries,
      align=center
    ] at (current page.center) {
        {\fontsize{60}{70}\selectfont WORKING DRAFT} \\ [0.8cm]
        {\Huge \DTMnow} \\
        {\Huge doi.org/10.17605/OSF.IO/EV8H6} \\ [0.8cm]
        {\Large Computational Complex Systems Laboratory} \\
        {\Large Homey Music, Detroit, USA} \\ [0.2cm]
         \includegraphics[width=1.0cm]{logo.png} 
    };
  \end{tikzpicture}
}

% -----------------------------------------

\DeclareMathOperator*{\argmin}{argmin}
\begin{document}

\title{Quantum Phase-Space from Information Erasure}

\begin{abstract}
We propose a derivation of quantized geometry by applying Landauer’s principle to the symplectic quantum of action within a thermodynamic framework. By maximizing entropy through information erasure within the minimal Heisenberg cell, we show that the continuum collapses into a discrete phase space. Numerical analysis demonstrates that this geometry accurately reproduces key quantum phenomena: Wigner quasiprobability marginals, black hole quasinormal modes, and Bell-violating spin correlations. This work suggests that quantum phase space arises as a geometric consequence of an entropic bound on symplectic action.
\end{abstract}

\author{Brian S. Mulloy\orcidlink{0000-0002-1803-3172}}
\email{brian@homeymusic.com}
\affiliation{Computational Complex Systems Laboratory\\Homey Music, Corktown, Detroit, MI, USA}
\date{\today}

\maketitle

\section{Overview}

Total system entropy $\mathcal{S} = \mathrm{H} + \mathrm{K}$ scales with the system's action $\mathcal{A}$ normalized by the quantum of action $A_0 = \frac{\pi}{4} \Delta q \Delta p = \frac{h}{2}$. Here, $\mathrm{H}$ and $\mathrm{K}$ represent Shannon entropy and Kolmogorov complexity, respectively. 

\begin{equation}
\mathcal{S} \approx C(n) \left( \frac{\mathcal{A}}{A_0} \right)^{\frac{n_\perp}{2}}
\end{equation}

Where $n$ is the physical spatial dimension and $n_{\perp}$ is the transverse reduction. The geometric coefficient $C(n)$ accounts for the symplectic projection of the $n$-dimensional space onto the information horizon $n_\perp$. For both the 1D QHO and tomographic probability distributions of spin, $n=1$, $n_{\perp} = 1/2$, and $C(1) = \frac{\pi}{4}$; whereas for the 3D Schwarzschild Horizon, $n=3$, $n_{\perp} = 2$, and $C(3) = \frac{\pi}{6}$.

\begin{figure}[htbp]
    \centering
    \includegraphics[width=\linewidth]{erase.pdf}
    \label{fig:quantum_blob}
    \caption{\textbf{Erasure within the minimal Heisenberg cell} (schematic). Within the symplectic quantum of action $\delta_q < q_0 : A_0 = \frac{\pi}{4} q_0 p_0 = \frac{h}{2}$, all surplus bits are erased from the thermodynamically inaccessible microstate $q_?$ specified by a program (e.g., $s_? = 110101010\dots$), collapsing to the macrostate $\mathrm{Q}$ with the minimal program (e.g., $\mathbf{s}^* = 1101$) of length $|\mathbf{s}^*|$ that runs on a universal computing machine $\mathcal{U}_{\operatorname{L}}$. This Landauer reduction constitutes a work cost of $W \ge k_B T \ln(2) K(Q)$, where $K(Q) = |\mathbf{s}^*|$ is the Kolmogorov complexity of the physical state.}
\end{figure}

\begin{figure}[htbp]
    \centering
    \includegraphics[width=\linewidth]{zurek_entropy.pdf}
    \caption{\textbf{Total system entropy versus momentum with transitions highlighted.}}
    \label{fig:zurek_entropy}
\end{figure}

\begin{figure}[htbp]
    \centering
    \includegraphics[width=\linewidth]{distributions.pdf}
    \caption{\textbf{Quantum harmonic oscillator coordinate distributions before, at, and after entropy phase transitions.}}
    \label{fig:distributions}
\end{figure}


\begin{figure}[htbp]
    \centering
    \includegraphics[width=\linewidth]{kolmogorov_complexity.pdf}
    \caption{\textbf{Kolmogorov complexity versus momentum with transitions highlighted.}}
    \label{fig:kolmogorov_complexity}
\end{figure}

\begin{figure}[htbp]
    \centering
    \includegraphics[width=\linewidth]{shannon_entropy.pdf}
    \caption{\textbf{Shannon entropy versus momentum with transitions highlighted.}}
    \label{fig:shannon_entropy}
\end{figure}

\section{Notation}
The variables used primarily throughout this manuscript are dimensionless values—such as coordinate $\mathcal{L}$, momentum $\mathcal{P}$, energy $\mathcal{E}$, and time $\mathcal{T}$—normalized to their characteristic base units ($q_0, p_0, E_0, t_0$). We define the fundamental \textit{quantum of action} as $A_0 = \frac{\pi}{4} q_0 p_0 = \frac{h}{2}$, which establishes the phase-space resolution of the system. While the base units ($q_0, p_0$) define the system's characteristic scales, the dynamic state resolutions ($\Delta q, \Delta p$) are captured by the dimensionless parameters $\mathcal{L}$ and $\mathcal{P}$.

Physical quantities are normalized as follows:

\noindent
\textbf{Normalized Notation} \\
Microstate $q, p$: $\mu_q = q_0 q_?, \quad \mu_p = p_0 p_?$ \\
Macrostate $q, p$: $q = \mathrm{Q} q_0, \quad p = \mathrm{P} p_0 \quad (\mathrm{Q, P} \in \mathbb{Q})$ \\
Coordinate: $q(t) = q_0 \mathcal{L}(t)$ \\
Momentum: $p(t) = p_0 \mathcal{P}(t)$ \\
Time: $t = t_0 \mathcal{T}$ \\
Energy: $E = E_0 \mathcal{E}$ \\
Quantum of Action: $A_0 = \frac{\pi}{4} q_0 p_0 = \frac{h}{2}$ \\
Normalized Action: $\mathcal{A} = \mathcal{P}^2 + \mathcal{L}^2$ \\
Action: $A = A_0 \mathcal{A}$ \\
Hamiltonian: $\mathcal{H} = A f = E_0 \mathcal{A} = \frac{h}{2}\left(\mathcal{P}^2 + \mathcal{L}^2\right)f$

\noindent
\textbf{System Operators and States} \\
Microstate (Initial): $\mu_q := q_? q_0 $ \\
Dimensionless Microstate (Initial): $q_?$ \\
Macrostate (Final): $\mathrm{Q} \in \mathbb{Q}$ \\

\noindent
\textbf{Universal Computing Machine} \\
Landauer Universal Computing Machine: $\mathcal{U}_{\operatorname{L}}$ \\
Programs Bit String: $s^*$ \\
Program length: $|s^*|$ \\
Bounded by the available action $S \le A/A_0$\\
Physical Entropy: $\mathrm{S}= \mathrm{H} + \mathrm{K}$ \\
Shannon Entropy: $\mathrm{H}$ \\
Kolmogorov Complexity: $\mathrm{K} = |s^*|$\\
Erasure: $\mathcal{E}$, \quad $\Omega(\mathrm{Q}) = |\{q_?\} \overset{\mathcal{E}}{\mapsto} \mathrm{Q}|$\\
Momentum: \(p=p_{0}\mathcal{P}\), \ \small(\(\mathcal{P}\): momentum cardinality) \\
Criteria: $|q_?-\mathrm{Q}| < \frac{\pi}{4} \mathcal{P}^{-1}$ \\

\noindent
\textbf{Energy and Informational Quantities} \\
Hamiltonian: $\mathcal{H}$ \\
\\
\noindent
\textbf{Geometric and Action Scales} \\
Action: $\mathcal{A}$ \\
Quantum of Action: $\mathcal{A}_0$ \\
Action Coordinate Constraint: $\Lambda$ \\
Action Momentum Pressure: $\Pi$

\begin{equation}
\delta_q \Delta p < \frac{h}{2}
\end{equation}
\begin{equation}
\Delta q \delta_p < \frac{h}{2}
\end{equation}

\section{Notation}

\subsection*{Informational Substrate and the Universal Computer}

In defining the bit-length $|s^*|$ of a physical state, we must establish the computational substrate upon which these specifications are constructed. Consistent with Zurek's observation that the selection of a specific universal computer $\mathcal{U}$ is mathematically arbitrary for the determination of algorithmic complexity, we adopt a framework scaled to the fundamental units of phase space.

\subsection*{The Stern-Brocot Computer}
Throughout this paper, we utilize a universal computer $\mathcal{U}$ based on the \textbf{Stern-Brocot construction}, with its output scaled to the characteristic Planck-scale unit $x_0$:
\begin{equation}
\mathcal{U} := x_0 \operatorname{SB}
\end{equation}
By defining the computer in this manner, we establish a direct isomorphism between the physical resolution limit $1/\mathcal{P}$ and the algorithmic depth of the coordinate specification. The Hamiltonian's action budget $S$ thus dictates the maximum depth of the Stern-Brocot path, ensuring that the bit-length $|s^*|$ remains physically bounded by the symplectic floor.

\subsubsection*{The Invariance of the Computational Substrate}
While the specific bit-length of a coordinate is dependent on the selection of the universal computer $\mathcal{U}$, the validity of the measure rests entirely on the consistency of the chosen substrate. As established by Li and Vit\'{a}nyi \cite{li_vitanyi_2026} and emphasized by Zurek \cite{zurek_1989}, any comparative analysis of bit-strings $s$ remains objective so long as all strings are executed on the same reference machine. 

This consistency addresses the objection that highly symmetric irrational values, such as the golden ratio ($\phi$), could be "compressed" into a trivial program (e.g., "repeat 10"). While one is free to imagine a symbolic computer that outputs $\frac{1+\sqrt{5}}{2}$ as a single primitive—a device whose name was Euclid—such a computer represents a different physical mapping. Within the fixed framework of $\mathcal{U}_{SB}$, every bit-string $s$ is parsed strictly as a path to a rational terminus $X \in \mathbb{Q}$ that saturates the Heisenberg cell. Because we maintain $\mathcal{U}_{SB}$ as our invariant reference, the bit-length $|s^*|$ provides a consistent and objective measure of the Hamiltonian action required for state specification.


\textit{Note: For a formal treatment of the Stern-Brocot tree as an algorithmic complexity generator and its mapping to rational coordinate approximations, see Appendix A.}

\section*{Symplectic Reciprocity and Conjugate Specification}

While the derivations in this paper primarily focus on the specification of the position coordinate $x$ relative to the momentum scaling factor $\mathcal{P}$, the theory maintains strict \textbf{Symplectic Reciprocity}. 

\subsection*{The Symmetry of Bit-Allocation}
In a symplectic phase space, the informational budget provided by the Hamiltonian $H$ is a shared resource between conjugate pairs. The resolution of position is "paid for" by the action capacity of momentum, and vice versa:
\begin{equation}
|s_x^*| \propto \mathcal{P} \quad \text{and} \quad |s_p^*| \propto \mathcal{L}
\end{equation}
To avoid redundant formalism, we define the \textbf{Conjugate Bound}: any algorithmic result derived for the specification of $x$ using the computer $\mathcal{U} := x_0 \operatorname{SB}$ holds isomorphically for the specification of $p$ using the conjugate computer $\mathcal{U}^\dagger := p_0 \operatorname{SB}$.

\subsection*{The Total Informational Volume}
The fundamental limit of the theory is not on a single coordinate, but on the \textbf{Total Specification}:
\begin{equation}
|s_x^*| + |s_p^*| \leq \frac{S}{S_0}
\end{equation}
This ensures that any "gain" in bit-length for the position coordinate necessitates a corresponding "erasure" or lack of specification in the momentum coordinate, preserving the hard symplectic boundary $h/2$. Throughout the following sections, coordinate-specific results should be understood as projections of this unified 2D informational constraint.


\subsection*{Inaccessible Microstates}

We define $x_{?}$ as an unknowable value, a pre-erasure microstate: the original, fine-grained coordinate of the system. We say that information has been erased during an observation because we will never find out what the state $x_{?}$ was originally. Inside the quantum of action $S = x_0 p_0 = h/2$ (the coarse-grained minimal Heisenberg cell), $x_{?}$ is operationally indistinguishable from the resulting observed state $x$ such that 
$$\frac{\pi}{4}\mathcal{L}|q_{?}-\mathrm{Q}|\Delta p<h/2.$$ 

Because the specific value of $x_{?}$ is rendered forever unknowable by this many-to-one compression, the erasure is logically irreversible. 

 The difference between $x$ and $x_{?}$ is that the algorithmic complexity $|b_?|$ of $x_?$ is greater than the algorithmic complexity of the minimal algorithmic complexity value $|b_?| > |b^*|$ within the quantum of action.

\medskip

We define $b$ as the program bit string of length $|b|$ required to specify 
the observed state $x$. Because every additional bit in the string incurs 
a thermodynamic erasure cost of $kT \ln 2$, the physical system is subject 
to an algorithmic-thermodynamic pressure. It is therefore a thermodynamic 
necessity that from the ensemble of available microstates 
$\{x_? : |x_? - x| \Delta p < h/2 \}$, the observed value $x$ must 
correspond to the unique bit string $\mathbf{b^*}$ of minimal length 
$|\mathbf{b^*}|$. This is expressed as the algorithmic mapping 
$$x = \mathcal{U}(\mathbf{b^*})$$ 
where $\mathcal{U}$ represents the universal physical constructor that 
identifies the state $x$ as the output of the minimal program $\mathbf{b^*}$ 
that satisfies the resolution limit $|x_{?} - x| \Delta p < h/2$. In this 
framework, the transition to a quantized state is a thermodynamic selection of the 
lowest-complexity representation permitted by the system's energy scale. 
This selection is governed by the principle that the universe naturally 
settles into the state of \textit{minimal informational action} [1]. As 
Landauer established, longer strings represent higher-potential energy states 
for the coupled system-reservoir ensemble, and thus $\mathbf{b^*}$ is the 
only state whose entropy can be fully dissipated by the local thermal 
background [2]. Furthermore, in the framework of the 
\textit{Foundations of Algorithmic Thermodynamics} [3], $\mathbf{b^*}$ 
represents the "coarsest" algorithmic descriptor that remains invariant 
under environmental interaction. Any higher-complexity representation $b_{n}$ 
is thermodynamically unstable; the extra information is effectively 
"scrambled" by the thermal reservoir, forcing the system to halt at the 
minimal bit string $\mathbf{b^*}$. It is this thermodynamic necessity of 
erasing the original state and selecting the shortest description 
$\mathbf{b^*}$ that forces the phase-space to emerge as a quantized 
structure rather than a continuous one.

\medskip

The implementation of the universal constructor $\mathcal{U}$ is operationally realized through the Stern-Brocot decoding function $\operatorname{SB}$, which maps the minimal program $\mathbf{b^*}$ to a unique rational $X \in \mathbb{Q}$. The constructor is thus defined by scaling this rational address with the fundamental length unit:
\begin{equation}
\mathcal{U}(\mathbf{s^*}) = x_0 \operatorname{SB}(\mathbf{s^*})
\end{equation}
In this framework, the Stern-Brocot tree—a complete binary search tree of all positive rationals—acts as the decoding architecture for phase-space. The minimal program $\mathbf{b^*}$ is an ordered bit string where each bit represents a discrete branching instruction: a '0' for a left (L) traversal and a '1' for a right (R) traversal. The constructor $\mathcal{U}$ reads these instructions to navigate from the root toward the rational $X$ that identifies the quantized state.
\begin{equation}
x = x_0 \operatorname{SB}(\mathbf{b^*}) = x_0 X
\end{equation}
This traversal is not indefinite; it is governed by the thermodynamic necessity to halt. As the tree is descended, the bit-length $|\mathbf{b^*}|$ increases, incurring an cumulative erasure cost of $|\mathbf{b^*}|kT\ln 2$. Consequently, the algorithmic thermodynamics select the exact level where the resolution limit $|x_{?} - x_0 \operatorname{SB}(\mathbf{b^*})| \Delta p < h/2$ is first satisfied. By embedding the Stern-Brocot tree directly into the constructor, the physical system ensures that the resulting quantized value $x$ is the most algorithmically efficient representation of the  available microstates 
$\{x_? : |x_? - x_0 X| \Delta p < h/2 \}$. This mechanism minimizes the complexity of the physical record by selecting the most accessible rational convergent within the resolution bound, effectively forcing the phase-space to emerge as a quantized structure from the thermodynamic limit of information erasure.

\medskip
The physical impossibility of resolving irrational states is best illustrated by the Golden Ratio, 
$$\phi \approx 1.618 \ 033 \ 988 \ 749 \ \dots$$ 
In the Stern-Brocot implementation of $\mathcal{U}$, the program required to compute $x = \phi x_0$ is a non-terminating instruction set $b_{\phi} = (1, 0, 1, 0, \dots)$. This infinite string exposes a critical divergence between the Shannon entropy $H_{Sh}$---the \textit{ensemble average} of information---and the algorithmic entropy $H_{alg}$, which quantifies the physical complexity of the \textit{individual microstate}. 

Following the framework of algorithmic thermodynamics \cite{Ebtekar2025}, we define the relationship between the ensemble average and the individual microstate complexity:
\begin{equation}
    H_{Sh} = -\sum_{i \in \{0,1\}} w_i \log_2 w_i, \quad H_{alg}(x_{?}) = |b|
\end{equation}
where $w_i$ represents the statistical probability of a branching instruction $i$ within the ensemble, and $H_{alg}(x_{?})$ is the actual length of the instruction set $b$ required to manifest the specific microstate $x_?$. While the Shannon entropy of the alternating rule for $\phi$ remains finite and low, the physical Landauer cost of manifesting the individual microstate is driven by the absolute bit-length of its program:
\begin{equation}
    \mathcal{W} = \lim_{|b| \to \infty} |b| k T \ln 2 = \infty
\end{equation}

This distinction is fundamental: whereas Shannon entropy measures our mean ignorance of an ensemble, the algorithmic complexity $|b|$ quantifies the actual thermodynamic weight of the specific microstate at each step of its resolution. As Zurek established in his formulation of \textit{physical entropy} \cite{Zurek1989}, the total entropy $S_d = H_{alg} + H_{Sh}$ must account for the information already manifested in the record. Because every additional bit in $b$ incurs a cumulative energy cost, the physical system is subject to an informational pressure that favors the state of minimal complexity. Since no local thermal reservoir possesses the infinite energy capacity required to finalize the non-terminating program $b_\phi$, the constructor $\mathcal{U}$ is forced to \textit{halt} at the minimal program $\mathbf{b^*}$---the unique rational convergent that satisfies the \textbf{Heisenberg resolution limit} $|x_{?} - x| \Delta p < h/2$ with the least thermodynamic work. This thermodynamic barrier proves that irrational coordinates are physically unrealizable, leaving the quantized rational record as the only possible state of the phase-space.


\subsection{Statistical Heisenberg Uncertainty Principle}

$$
\sigma_x \sigma_p \ge \frac{\hbar}{2}
$$

\subsection{Symplectic Heisenberg Uncertainty Principle}

\subsubsection{The Quantum of Action}

$$
S_0 = x_0 p_0 = \frac{h}{2}
$$

\subsubsection{The Symplectic Capacity}

$$
S = S_0 ( \mathcal{L} \times \mathcal{P}) \ge \frac{h}{2}
$$

$$
S = c(\Omega)
$$

where $L$ and $P$ are dimensionless scaling factors of the characteristic position $x_0$ and momentum $p_0$.

$$
\Delta x \Delta p \ge \frac{h}{2}
$$

$$
\Delta x := \mathcal{L} x_0 \quad \Delta p := \mathcal{P} p_0
$$

\subsection{Quantum Harmonic Oscillator}

$$
H = \frac{\mathcal{P}^2 p_0^2}{2m} + \frac{1}{2}m\omega^2 \mathcal{L}^2 x_0^2
$$

$$
H = \frac{h}{2}(\mathcal{P}^2 + \mathcal{L}^2)f
$$

\subsection{Fundamental Fraction}


\begin{quote}

We say that information has been erased during the compression because we will never find out where the molecule was originally.

\hfill --- \textit{The Physics of Forgetting}

\end{quote}

We say that information has been erased $|\mathbf{s^*}|kT\ln{2}$ when the microstate collapses to the macrostate $q_? \rightarrow \mathrm{Q}, \ \mu_q = q_?q_0, \ q =  \mathrm{Q} q_0, \ \mathrm{Q} \in \mathbb{Q}$ (dimensionless, normalized units) because we will never find out what the state $q_?$ was originally. State values within the coarse-grained Heisenberg uncertainty cell (quantum blob) are operationally indistinguishable 
$\frac{\pi}{4}q_0\mathcal{L}|q_? - \mathrm{Q}| \Delta p < h/2$. Each $\mathrm{Q}$ corresponds to a bit string $\mathbf{s^*}$ which is unique for the given physical system. For any physical system $\Delta q := q_0\mathcal{L}, \ \Delta p := p_0 \mathcal{P}$ is defined by the Hamiltonian. The dimensionless normalized factor $\mathcal{P}$ indicates the momentum and also the resolution limit of the conjugate observable and is determined by the Hamiltonian's energy scale. We find $\mathrm{Q}$ and $\mathbf{s^*}$ using the Stern-Brocot tree such that $\delta_q = |q_?/q_0 - \mathrm{Q}| < 1 / \mathcal{P} \implies \delta_q \Delta p < h/2$, using the hard symplectic boundary $h/2$ of the Heisenberg cell as opposed to the fuzzy statistical one $\hbar/2$. The spread of the position and momentum as $\mathcal{P}$ increases the ensemble Shannon entropy $\mathrm{H}$ while the increasing resolution $1/\mathcal{P}$ increases the algorithmic entropy $\mathrm{K}$ required to erase the information about the unknown state.

$$
|q_? - \mathrm{Q}| < \Delta q
\implies
\frac{\pi}{4}\mathcal{L}|q_? - \mathrm{Q}| \Delta p < \frac{h}{2}
$$


$$
\delta_q = 
\frac{|x_? - x|}{x_0} = 
\frac{\left | x_? - \frac{a}{b} x_0 \right |}{x_0} = 
x_0 \frac{\left | \frac{x_?}{x_0} - \frac{a}{b} \right | }{x_0} =
\left | \frac{x_?}{x_0} - \frac{a}{b} \right |
$$

$$
\left | \frac{x_?}{x_0} - \frac{a}{b} \right | x_0 p_0 P < S
$$

To satisfy the containment condition $|x_?-x|<\Delta x$ with minimal algorithmic overhead, the search for a rational approximation must saturate the symplectic floor. At this limit of information erasure, the coarse-grained action converges to the quantum of action: $S\rightarrow S_0=h/2$.

$$
\left | \frac{x_?}{x_0} - \frac{a}{b} \right | S_0 P < S_0
$$


$$
\left | \frac{x_?}{x_0} - \frac{a}{b} \right | \frac{h}{2} P < \frac{h}{2}
$$

$$
\delta_x = \left | \frac{x_?}{x_0} - \frac{a}{b} \right | < \frac{1}{P}
$$

We have a dimensionless form of Zurek's equation $\delta_x = 1/P$.

\subsection{Symplectic Symmetry}
The derivation for the conjugate variable $p$ follows by the same logic:
$$ \delta_p = \frac{|p_? - p|}{p_0} = \left | \frac{p_?}{p_0} - \frac{c}{d} \right | $$
Applying the same boundary condition for the blob area:
$$ \delta_p p_0 \Delta x \ge \frac{h}{2} \implies \left | \frac{p}{p_0} - \frac{c}{d} \right | x_0 p_0 L \ge \frac{h}{2} $$
Substituting the fundamental action unit $x_0 p_0 = h/2$ yields the symmetric result:
$$ \delta_p = \left | \frac{p_?}{p_0} - \frac{c}{d} \right | \ge \frac{1}{L} $$
We have a dimensionless form of Zurek's equation $\delta_p = 1/L$.
This demonstrates that the resolution of phase space is constrained by the same algorithmic entropy in both coordinates.

\subsection{On Sub-Planck Scales}
We suggest that the sub-Planck structures identified by Zurek are not violations of the fundamental quantum of action, but rather a manifestation of symplectic squeezing. By allowing the quantum blob $a=h/2$ to evolve into an elliptical geometry defined by our scaling factors $L$ and $P$, one can recover Zurek’s high-resolution interference features. In this view, the sub-Planck scale actually represents a one-dimensional resolution limit $\delta_{x}$ or $\delta_{p}$ of a squeezed symplectic invariant, rather than a reduction of the total phase-space area below the de Gosson boundary.

\section*{Informational Action Bound Theory}

\subsection*{Context: The Complexity-Action (CA) Conjecture}

The proposal that physical specification is bounded by action finds significant precedent in the \textbf{Complexity-Action (CA) Conjecture}, a major hypothesis in high-energy physics and AdS/CFT duality proposed by Susskind, Brown, and colleagues.

\subsection*{The Central Hypothesis}
The CA conjecture posits that the quantum complexity ($\mathcal{C}$) of a boundary state is exactly proportional to the classical action ($S$ or $\mathcal{I}$) of a specific spacetime region known as the Wheeler-DeWitt patch. This is formally expressed as:
\begin{equation}
\mathcal{C} = \frac{\mathcal{I}}{\pi \hbar}
\end{equation}

\subsection*{Relevance to the Coordinate-Action Bound}
The CA conjecture provides a direct academic foundation for the claim that "one cannot specify the value of any physical state with a bit length greater than the capacity of the action." It implies that action is not merely a dynamical integral, but the physical manifestation of algorithmic complexity. 

While the CA conjecture typically applies to holographic boundary states, the \textbf{Coordinate-Action Bound} extends this logic to the minimal bit-length specification of individual phase-space coordinates ($x, p$), identifying the symplectic quantum $h$ as the fundamental resolution limit for such physical information.

\section*{The Minimal Action-Complexity Bound}

This section summarizes the fundamental limit of physical specification. The theory posits that the algorithmic complexity of a coordinate is not an abstract property, but a physical value whose existence is "funded" by the action capacity of the system.

\subsection*{The Nutshell Expression}
For any physical state defined by coordinates $(x, p)$ within a system governed by a Hamiltonian $H$, the bit-length $|s^*|$ of the shortest description of that state is strictly bounded by the normalized action capacity:

\begin{equation}
\forall (x,p) \in H, \quad |s^*| \leq \frac{S}{S_0}
\end{equation}

\subsection*{Definitions and Terms}
\begin{itemize}
    \item \textbf{$|s^*|$}: The minimal bit-length (Kolmogorov complexity) required to specify the coordinate. In this framework, this corresponds to the depth of the Stern-Brocot path needed to reach the rational approximation $X = a/b$.
    \item \textbf{$S$}: The symplectic action (capacity) provided by the Hamiltonian $H$.
    \item \textbf{$S_0$}: The symplectic quantum of action, defined as $S_0 = h/2$. This represents the "cost" of a single unit of information.
    \item \textbf{$H$}: The Hamiltonian, which acts as the physical "bit-budget" provider. The energy scale of $H$ determines the scaling factors $\mathcal{L}$ and $\mathcal{P}$, which in turn dictate the maximal resolution.
\end{itemize}

\subsection*{The Physical Mechanism}
As demonstrated in the derivation of the dimensionless Zurek equation ($\delta_x = 1/\mathcal{P}$), the Hamiltonian's energy scale dictates the momentum scaling $\mathcal{P}$. This scale determines the maximal resolution $1/\mathcal{P}$ of the position coordinate. 

The search for a rational approximation on the Stern-Brocot tree must saturate the symplectic floor to satisfy the containment condition $|x_? - x| < \Delta x$. Consequently, the bit-string length $|s^*|$ is physically prohibited from exceeding the ratio $S/S_0$. Information beyond this bound is not merely unknown; it is physically non-existent as it lacks the action substrate required to define it.

\section*{Derivation of the Informational Hamiltonian}

The bit-budget of a physical coordinate is derived directly from the energy scales of the system. We demonstrate this by transforming a traditional Hamiltonian into its dimensionless form, where energy is expressed as the product of frequency and action.

\subsection*{1. The Traditional Hamiltonian}
Consider the standard Hamiltonian for a Quantum Harmonic Oscillator (QHO):
\begin{equation}
H = \frac{p^2}{2m} + \frac{1}{2}m\omega^2 x^2
\end{equation}

\subsection*{2. Introduction of Characteristic Scales}
We define characteristic units $x_0$ and $p_0$ to satisfy the ground-state equilibrium where kinetic and potential energies are equal. In the symplectic convention ($S_0 = h/2$), these units are:
\begin{equation}
x_0 = \sqrt{\frac{h}{2m\omega}}, \quad p_0 = \sqrt{\frac{hm\omega}{2}}
\end{equation}
Substituting the dimensionless scaling factors $\Delta x = \mathcal{L} x_0$ and $\Delta p = \mathcal{P} p_0$:
\begin{equation}
H = \frac{\mathcal{P}^2 p_0^2}{2m} + \frac{1}{2}m\omega^2 \mathcal{L}^2 x_0^2
\end{equation}

\subsection*{3. The Dimensionless Form}
To pull out the frequency $f = \omega/2\pi$, we perform the following substitution steps:
\begin{itemize}
    \item \textbf{Kinetic Substitution:} $\frac{\mathcal{P}^2 (hm\omega/2)}{2m} = \mathcal{P}^2 \frac{h\omega}{4}$
    \item \textbf{Potential Substitution:} $\frac{1}{2}m\omega^2 \mathcal{L}^2 (\frac{h}{2m\omega}) = \mathcal{L}^2 \frac{h\omega}{4}$
    \item \textbf{Factoring Frequency:} Recalling $\omega = 2\pi f$, both terms share the coefficient $\frac{h(2\pi f)}{4} = \frac{h}{2} \pi f$. 
\end{itemize}
Under the symplectic normalization where the angular factor is absorbed into the coordinate scaling, the Hamiltonian simplifies to:
\begin{equation}
H = \frac{h}{2}(\mathcal{P}^2 + \mathcal{L}^2)f
\end{equation}

\subsection*{4. Mapping Action to Bit-Length}
The scaling factor $\mathcal{P}$ defines the momentum-driven resolution of the conjugate coordinate ($1/\mathcal{P}$). Because the bit-length $|s^*|$ is the depth of the Stern-Brocot tree required to reach this resolution:
\begin{equation}
|s_x^*| \approx \log_2(\mathcal{P}) \quad \text{subject to} \quad \frac{h}{2}(\mathcal{P}^2 + \mathcal{L}^2)f \leq E
\end{equation}

\subsection*{Conclusion}
The Hamiltonian energy $E$ limits the scaling factors $\mathcal{L}$ and $\mathcal{P}$. Since these factors dictate the maximal path-length on the Stern-Brocot tree, the bit-length of any coordinate is physically bounded by the system's total action:
\begin{equation}
|s^*| \leq \frac{S}{S_0}
\end{equation}

\section*{The Coordinate-Action Bound}

This theory posits that a physical coordinate ($x$ or $p$) is an informational entity whose resolution is strictly bounded by the action capacity of its governing Hamiltonian. The value of any coordinate cannot be specified with a bit-length $|s^*|$ greater than the capacity provided by the fundamental quantum of action.

\subsection*{The Unit of Specification}
We define the symplectic quantum of action as $S_0 = h/2$. This constant represents the minimal physical substrate required to support a single unit of coordinate specification. 

\subsection*{Coordinate-Specific Bounds}
For the $x$ and $p$ coordinates in phase space, the bit-lengths $K(x) = |s_x^*|$ and $K(p) = |s_p^*|$ are bounded by the ratio of the coordinate-associated action to the unit quantum:
\begin{equation}
|s_x^*| \leq \frac{S_x}{S_0} \quad \text{and} \quad |s_p^*| \leq \frac{S_p}{S_0}
\end{equation}
Here, the ratio $\frac{S_p}{S_0}$ is not merely a scaling factor; it represents the total \textbf{bit-budget} available for that coordinate. Once the physical action $S_p$ is exhausted, no further information (bit-length) can be attributed to the value of $p$.

\subsection*{Hamiltonian Information Capacity}
The total information content of the state $(x, p)$ is constrained by the integrated action of the Hamiltonian $H$ over the time interval $\Delta t$:
\begin{equation}
|s_x^*| + |s_p^*| \leq \frac{1}{2S_0} \int_{t_0}^{t_1} H(x, p) \, dt
\end{equation}
In this framework, physical action is the currency of information. The "knowability" of a coordinate is a finite resource governed by the energy-time volume of the system.
d

\section{Notes}
1. Margolus-Levitin Theorem and the "Quantum Speed Limit" The most direct parallel is the Margolus-Levitin Theorem, which states that the time \(\tau \) to reach an orthogonal state is bounded by \(\tau \ge \frac{h}{4E}\). If we rearrange this as \(\frac{E\tau }{h}\ge \frac{1}{4}\), and recognize \(E\tau \) as a form of action (\(S\)), the theorem essentially says that you need at least a quarter-unit of action to perform "one bit" of state-change.Recent 2025 refinements to this theorem have extended these bounds to "arbitrary fidelity," confirming that the number of operations per second is strictly limited by the system's energy.

2. Honoring Brocot: The "erasure" \(\mathcal{E}(x_?) = x\) is the process of several teeth mapping to the same "tick" of the clock.
\section*{Unification of Symplectic Erasure and the Holographic Scaling of Action}

\subsection*{Foundational Framework}
We model the emergence of quantum macrostates from the discrete erasure of hidden microstates $\mu_q$ within a phase-space manifold. Following \textbf{de Gosson's} framework, the fundamental unit of phase space is the \textit{quantum blob}: a symplectic invariant with finite support and capacity $A_0 = h/2$. Unlike the traditional HUP, which allows for infinite-tailed distributions, the quantum blob enforces a \textbf{rigid topological boundary} that limits information density.

\subsection*{The Generalized Scaling Law}
The total system entropy $\mathcal{S} = \mathrm{H} + \mathrm{K}$ (Shannon entropy and Kolmogorov complexity) is bounded by the ratio of the total Hamiltonian action $\mathcal{A}$ to the quantum of action $A_0$, governed by the dimensionality of the \textbf{information horizon}:

\begin{equation}
\mathcal{S} \approx C(n) \left( \frac{\mathcal{A}}{A_0} \right)^{\frac{n_\perp}{2}}
\end{equation}

\noindent Where:
\begin{itemize}
    \item $n$ is the physical spatial dimension.
    \item $n_\perp = n - 1/2$ is the \textbf{transverse reduction index}, representing the dimensional mismatch between the bulk action and the erasure gate.
    \item $C(n) = V_n / 2^n$ is the \textbf{geometric coefficient} (packing efficiency), representing the ratio of a unit $n$-ball to its bounding hypercube.
\end{itemize}

\subsection*{Dimensional Hierarchy and the CA Limit}
\begin{table*}[t]
\centering
\begin{tabular*}{\textwidth}{@{\extracolsep{\fill}}lccccc}
\hline\hline
\textbf{Physical System} & \textbf{Spatial ($n$)} & \textbf{Transverse ($n_\perp$)} & \textbf{Coefficient $C(n)$} & \textbf{Growth Law} & \textbf{Information Regime} \\ \hline
1D QHO / Spin Tomography & 1 & 1/2 & $\pi/4$ & $\mathcal{S} \sim \mathcal{A}^{1/4}$ & Sub-Holographic \\
1D Light-Sheet / String  & 2 & 1   & $\pi/4$ & $\mathcal{S} \sim \mathcal{A}^{1/2}$ & Squeezed Camel \\
3D Schwarzschild Horizon & 3 & 2   & $\pi/6$ & $\mathcal{S} \sim \mathcal{A}^{1}$   & \textbf{Susskind-CA Limit} \\
5D De-Sitter Bulk        & 5 & 4   & $\pi^2/32$ & $\mathcal{S} \sim \mathcal{A}^{2}$ & Quadratic Complexity \\
\hline\hline
\end{tabular*}
\caption{The holographic scaling hierarchy. The 1D QHO is the most geometrically constrained case ($\mathcal{A}^{1/4}$), while the 3D horizon reclaims the linear \textbf{Complexity Equals Action} relationship. The discrepancy between $C(3) = \pi/6$ and the CA coefficient $2/\pi$ is attributed to the \textbf{finite support} of the de Gosson blob versus the infinite-tailed Gaussian assumptions of continuous HUP.}
\label{tab:CA_unification}
\end{table*}
\begin{quote}The discrepancy between the de Gosson geometric coefficient and the CA conjecture is the physical signature of \textbf{finite symplectic support}. Traditional HUP assumes infinite-tailed distributions (higher efficiency), whereas the emergent model accounts for the \textbf{topological packing density} of rigid quantum blobs, which limits information saturation to the volume ratio \(C(n)=V_{n}/2^{n}\). The coefficient (\(\pi /6\approx 0.524\)) is smaller than Susskind’s (\(2/\pi \approx 0.637\)) \end{quote}


\subsection*{Conclusion}
The $10^9$ data points obtained via \textbf{Stern-Brocot tree search} confirm that the algorithmic cost of erasure is a function of the \textbf{Symplectic Projection}. The shift from $\mathcal{A}^{1/4}$ in 1D to $\mathcal{A}^1$ in 3D represents the topological transition from a constrained "eye-of-the-needle" erasure to a fully saturated holographic horizon.

\end{document}
