\documentclass[%
 aps, reprint, prl, amsmath,amssymb
]{revtex4-2}

\usepackage{graphicx}% Include figure files
\graphicspath{{figures/}} 
\usepackage{dcolumn}% Align table columns on decimal point
\usepackage{bm}% bold math
\usepackage{orcidlink}
\usepackage{mathtools}

% -------------- Watermark ----------------
\usepackage[style=iso]{datetime2}
\usepackage{FiraMono} 
\usepackage{tikz}
\usepackage{transparent} 

% Use shipout/background to ensure it stays behind text/figures
\AddToHook{shipout/background}{
  \begin{tikzpicture}[remember picture, overlay]
    \node [
      rotate=55,
      scale=1.2,
      text opacity=0.15,      
      color=gray!50,          
      font=\fontfamily{FiraMono-TLF}\selectfont\bfseries,
      align=center
    ] at (current page.center) {
        {\fontsize{60}{70}\selectfont WORKING DRAFT} \\ [0.8cm]
        {\Huge \DTMnow} \\
        {\Huge doi.org/10.17605/OSF.IO/EV8H6} \\ [0.8cm]
        {\Large Computational Complex Systems Laboratory} \\
        {\Large Homey Music, Detroit, USA} \\ [0.2cm]
         {\transparent{0.25}\includegraphics[width=1.0cm]{logo.png}}
    };
  \end{tikzpicture}
}
% -----------------------------------------


\DeclareMathOperator*{\argmin}{argmin}
\begin{document}
\title{Information Erasure Inside the Quantum of Action}

\begin{abstract}
By maximizing entropy through Landauer information erasure inside de Gosson's quantum of action—the finite-support Heisenberg cell where states are operationally indistinguishable—we show that the continuum collapses into discrete conjugate observables. Consistent with the experimental difficulty in isolating pure states, numerical results confirm that as action increases, Schrödinger-eigenstates become vanishingly sparse configurations—not privileged solutions—among the expanding possibilities in phase space. At the entropic limit of action, the proposed model provides a thermodynamic-geometric correspondence to standard quantum mechanics.
\end{abstract}

\author{Brian S. Mulloy\orcidlink{0000-0002-1803-3172}}
\email{brian@homeymusic.com}
\affiliation{Computational Complex Systems Laboratory\\Homey Music, Corktown, Detroit, MI, USA}
\date{\today}

\maketitle

\section{Figures}

\begin{figure}[t!]
    \centering
    \includegraphics[width=\linewidth]{erase.pdf}
    \caption{\textbf{Information erasure inside conjugate action quanta} (schematic)}
\label{fig:erase}
\end{figure}

\section{Equations}

\subsection{The Information-Action Limit}

The action available for information erasure inside de Gosson's action quanta—the finite-support Heisenberg cells where states are operationally indistinguishable—is
\begin{equation}
\label{eq:InfoActionLimit}
b E \tau \le \pi \hbar ,
\end{equation}
where $b = b_q + b_p = |\mathbf{s}^*_q| + |\mathbf{s}^*_p|$ is the count of bits in the binary address $\mathbf{s}^*_q,\mathbf{s}^*_p \mapsto (q,p)$ supported by the available action for the physical state $(q,p)$, $E = k_B T \ln 2$ is the Landauer erasure energy, and $\tau$ is the time required to traverse to the binary address. Here $\pi \hbar = \frac{\pi}{4} \Delta q \Delta p$ is the de Gosson quantum of action, the minimal elliptical phase-space area of finite support; unlike statistical variance $\sigma_q \sigma_p$, these dimensions $\Delta q$ and $\Delta p$ define the hard Heisenberg boundaries where states are operationally indistinguishable \cite{DeGosson2003}.

\subsection{Information Erasure Inside the Conjugate Action Quanta}

Landauer erasure within a conjugate pair of Heisenberg cells is bounded by:
\begin{subequations}
\begin{equation}
\label{eq:SymplecticBoundsq}
\frac{\pi}{4}|\mu_q - q|\Delta p \le \frac{\pi \hbar}{2} 
\end{equation}
\begin{equation}
\label{eq:SymplecticBoundsp}
\frac{\pi}{4}|\mu_p - p|\Delta q \le \frac{\pi \hbar}{2} 
\end{equation}
\end{subequations}

where the operationally indistinguishable microstates $\{\mu_q, \mu_p\}$ are irreversibly erased to macrostates $(q,p)$, encoded with the action available in each orthogonal quanta, where the phase-space coordinates map to the binary addresses $(q,p) \mapsto\mathbf{s}^*_q,\mathbf{s}^*_p$.

\subsection{Thermodynamics and Geometry}

These equations relate by

\begin{equation}
\label{eq:ThermoGeo}
A_L \le A_{\texttt{erase}} \le A_d\Gamma = \pi \hbar,
\end{equation}

where $A_L \coloneqq b E \tau$, $A_{\texttt{erase}} = A_{\texttt{erase},q} + A_{\texttt{erase},p}$, $A_{\texttt{erase},q} \coloneqq \frac{\pi}{4}|\mu_q - q|\Delta p$ and $A_{\texttt{erase},p} \coloneqq \frac{\pi}{4}|\mu_p - p|\Delta q \le \frac{\pi \hbar}{2} $, and $A_d\Gamma \coloneqq \frac{\pi}{4} \Delta q \Delta p$. 

\subsection{Fine Structure}

The fine structure ratios are along the coordinate dimension from smallest to largest are the ratio of the erasure distance to the coordinate quantum of action bound (akin to $r_e/\lambda_e$)

\begin{equation}
\label{eq:alpha1}
\alpha = \frac{|\delta_q| \Delta p}{2 \hbar}
\end{equation} where $\delta_q \coloneqq \mu_q - q$ is the erasure distance and $\frac{2 \hbar}{\Delta p}$ is the quantum of action coordinate bound for $A_q$. The ratio of the coordinate quantum of action bound to the overall Hamiltonian (akin to $\lambda_e/a_0$)

\begin{equation}
\label{eq:alpha2}
\alpha = \frac{2 \hbar}{\Delta q \Delta p}.
\end{equation} And the ratio of the erasure distance to the overall Hamiltonian (akin to $r_e/a_0$)

\begin{equation}
\label{eq:alpha3}
\alpha^2 = \frac{\delta_q}{\Delta q}.
\end{equation}


\section{Normalized Equations}

\subsection{Information Erasure Inside the Quantum of Action}

Landauer erasure within a conjugate pair of Heisenberg cells is bounded by:
\begin{equation}
\label{eq:SymplecticBounds}
|q_? - \mathbf{Q}^*| \le \mathcal{P}^{-1}, 
\quad 
|p_? - \mathbf{P}^*| \le \mathcal{Q}^{-1},
\end{equation}
where the operationally indistinguishable microstates are represented as $\mu_q \coloneqq q_0 q_?$ and $\mu_p \coloneqq p_0 p_?$, while the macrostates $\mathbf{Q}^*$ and $\mathbf{P}^*$, encoded with the action available in each orthogonal quanta, are defined as $q \coloneqq q_0 \mathbf{Q}^*$ and $p \coloneqq p_0 \mathbf{P}^*$. The maximal extent and maximal momentum of the system are represented as $\Delta q \coloneqq q_0 \mathcal{Q}$ and $\Delta p \coloneqq p_0 \mathcal{P}$ 

\paragraph{Referencing Example:}
The symplectic squeezing is captured by Eq.~(\ref{eq:SymplecticBounds}).

\section{Foundational Manuscripts}

\begin{itemize}
    \item \textbf{Geometry}
    \begin{itemize}
        \item Heisenberg \cite{Heisenberg1927}
        \item de Gosson \cite{DeGosson2003}
    \end{itemize}
    
    \item \textbf{Thermodynamics}
    \begin{itemize}
        \item Landauer \cite{Landauer1961}
        \item Plenio and Vitelli \cite{PlenioVitelli2001}
        \item Margolus and Levitin \cite{MargolusLevitin1998}
    \end{itemize}
    
    \item \textbf{Methodology}
    \begin{itemize}
        \item Stern \cite{Stern1858}
        \item Brocot \cite{Brocot1861}
        \item Graham, Knuth, and Patashnik \cite{ConcreteMath}
        \item Stolzenburg \cite{Stolzenburg2015}
        \item Aiylam \cite{Aiylam2013}
    \end{itemize}
\end{itemize}

\bibliography{main} 
\end{document}
