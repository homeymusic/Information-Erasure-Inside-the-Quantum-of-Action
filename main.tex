\documentclass[%
 aps, reprint, prl, amsmath,amssymb
]{revtex4-2}

\usepackage{graphicx}% Include figure files
\graphicspath{{figures/}} 
\usepackage{dcolumn}% Align table columns on decimal point
\usepackage{bm}% bold math
\usepackage{orcidlink}
\usepackage{mathtools}

% -------------- Watermark ----------------
\usepackage[style=iso]{datetime2}
\usepackage{FiraMono} 
\usepackage{tikz}

\AddToHook{shipout/foreground}{
  \begin{tikzpicture}[remember picture, overlay]
    \node [
      rotate=55,              % Restores the diagonal aspect-ratio alignment
      scale=1.2,              % Large impactful scale
      text opacity=0.18, 
      color=gray!80,
      font=\fontfamily{FiraMono-TLF}\selectfont\bfseries,
      align=center
    ] at (current page.center) {
        {\fontsize{60}{70}\selectfont WORKING DRAFT} \\ [0.8cm]
        {\Huge \DTMnow} \\
        {\Huge doi.org/10.17605/OSF.IO/EV8H6} \\ [0.8cm]
        {\Large Computational Complex Systems Laboratory} \\
        {\Large Homey Music, Detroit, USA} \\ [0.2cm]
         \includegraphics[width=1.0cm]{logo.png} 
    };
  \end{tikzpicture}
}

% -----------------------------------------

\DeclareMathOperator*{\argmin}{argmin}
\begin{document}
\title{Information Erasure Inside the Quantum of Action}

\begin{abstract}
By maximizing entropy through Landauer information erasure within de Gosson's quantum of action, we show that continuum states collapse into discrete conjugate observables. Reflecting the experimental difficulty in isolating pure states, numerical results show that as action increases, Schrödinger-eigenstates become vanishingly sparse configurations—not privileged solutions—among the expanding possibilities in phase space. At the entropic limit of action, this work provides a thermodynamic-geometric correspondence to standard quantum mechanics.
\end{abstract}

\author{Brian S. Mulloy\orcidlink{0000-0002-1803-3172}}
\email{brian@homeymusic.com}
\affiliation{Computational Complex Systems Laboratory\\Homey Music, Corktown, Detroit, MI, USA}
\date{\today}

\maketitle

\section{Overview}

The characteristic action (facilitating dimensionless, normalized units) is defined as 
\begin{equation}
    A_0 = q_0 p_0 = \hbar.
\end{equation}

The de Gosson quantum of action, representing the minimum symplectic capacity of a phase-space ``blob,'' is defined by the area of an ellipse with conjugate diameters $\Delta q$ and $\Delta p$:
\begin{equation}
    A_{d\Gamma} = \frac{\pi}{4} \Delta q \Delta p = \pi \hbar.
\end{equation}

For a system with macroscopic semi-axes $\mathcal{Q}$ and $\mathcal{P}$, the total system action $A_\Gamma$ is the sum of these elementary quanta. Given that the system is defined by a conjugate pair of action-volumes, we account for both degrees of freedom in the counting logic:
\begin{equation}
    A_\Gamma = \sum A_{d\Gamma} = \pi \hbar \mathcal{Q} \mathcal{P} \ge \pi \hbar.
\end{equation}

The total available microstate capacity, or counting logic of the system, is thus:
\begin{equation}
    N_\Gamma = 2 \mathcal{Q}\mathcal{P}.
\end{equation}
For a macroscopic system, $N_\Gamma \gg 1$, whereas the ground state is reached at the bifurcation limit $N_\Gamma = 2$, representing the irreducible conjugate pair.
\section{The Thermodynamic Necessity of Conjugate Action Quanta}
\begin{figure}[htbp]
    \centering
    \includegraphics[width=\linewidth]{erase.pdf}
    \caption{\textbf{Erasure within the symplectic quantum of action} (schematic). Within the symplectic quantum of action defined by $A_{d\Gamma} = \frac{\pi}{4} \Delta q \Delta p = \pi \hbar$, the operationally indistinguishable microstates $\{q_?\}$ encoded as binary sequences $\{s_?\}$ (e.g., \texttt{11010101}) and bounded by the dimensionless conjugate $\pm 1/\mathcal{P}=\pm p_0/p$ are irreversibly erased by thermodynamic necessity. The indistinguishable microstates are collapsed to the macrostate $\mathbf{Q_0}\colon |q_? - \mathbf{Q_0}| = |\delta_q| \le \Delta q$ encoded as the minimal binary sequence $\mathbf{s_0}$ (e.g., \texttt{1101}) of length $K(\mathbf{Q_0}) = |\mathbf{s_0}|$, running on the universal erasing machine $\mathcal{U}_{\mathrm{L}}$. This erasure over a characteristic time $\tau_L$ requires an action $A_{\mathrm{L}} = \frac{\pi}{4} \delta_q \Delta p$ constrained from below by Landauer's thermodynamic principle and from above by de Gosson's geometric boundary $K(\mathbf{Q_0}, \mathbf{P_0}) \ln(2) k_{B} T \tau_L \le A_L \le \pi \hbar$.}
\label{fig:erase}
\end{figure}

\begin{figure}[htbp]
    \centering
    \includegraphics[width=\linewidth]{landauer_squeeze.pdf}
    \caption{\textbf{Landauer action versus de Gosson squeeze.}}
    \label{fig:landauer_squeeze}
\end{figure}

\begin{figure}[htbp]
    \centering
    \includegraphics[width=\linewidth]{distributions.pdf}
    \caption{\textbf{Quantum harmonic oscillator coordinate distributions before, at, and after Landauer action phase transitions.}}
    \label{fig:distributions}
\end{figure}

\begin{figure}[htbp]
    \centering
    \includegraphics[width=\linewidth]{conjugate_quanta.pdf}
    \caption{\textbf{Conjugate quanta of action.}}
    \label{fig:conjugate_quanta}
\end{figure}

\subsection{The Entropic Conflict}
The core constraint sets a strict cap on the thermodynamic complexity a single ``action blob'' can contain:
\begin{equation}
K(\mathbf{Q_0}, \mathbf{P_0}) \ln(2) k_{B} T \tau_L \le A_L \le \pi \hbar
\end{equation}
If one attempts to pack the full algorithmic information of both position and momentum into a single blob, the system exceeds the fundamental limit imposed by the Planck constant $\hbar$. A single quantum blob represents an absolute minimum symplectic volume, leaving no geometric ``slack'' for the simultaneous precision of conjugate values.

\subsection{The Additivity of Action-Information and the Conjugate Leap}

By defining the joint complexity of the system as the sum of independent binary programs on the universal Landauer machine $\mathcal{U}_{\mathrm{L}}$, we establish $K(\mathbf{Q_0}, \mathbf{P_0}) = |\mathbf{s_{Q_0}}| + |\mathbf{s_{P_0}}|$. Consequently, the Landauer bound on the characteristic erasure time $\tau_L$ is constrained by:
\begin{equation}
\left( |\mathbf{s_{Q_0}}| + |\mathbf{s_{P_0}}| \right) \ln(2) k_{B} T \tau_L \le A_L \le \pi \hbar
.
\end{equation}

This identity reveals that the ``leap'' from a single microstate continuum to a conjugate pair of observables is a thermodynamic necessity. Because the information budget is limited by the \textbf{symplectic capacity} \cite{deGosson2011} of the quantum blob ($\pi \hbar$), the system cannot resolve both $q$ and $p$ within a single quantum of action. 

To maintain the total action while reducing complexity through \textbf{Landauer erasure} \cite{Landauer1961}, the quantum of action bifurcates into a \textbf{conjugate pair} of action quanta:
\begin{itemize}
    \item \textbf{The Q-Blob:} Minimizes position uncertainty by utilizing the thermodynamic slack of the conjugate momentum program.
    \item \textbf{The P-Blob:} Minimizes momentum uncertainty by utilizing the thermodynamic slack of the conjugate position program.
\end{itemize}

This duality maps the bulk thermodynamic information into the discrete, erasable bits of our observable world, ensuring the erasure time adheres to the absolute \textbf{Planckian dissipation limit} $\tau_{\hbar} = \frac{\hbar}{k_B T}$, scaled by the geometric factor $\frac{\pi}{2\ln(2)}$.

\section{The Encoder}

To distinguish the creation of information from its execution on the universal machine $\mathcal{U}_L$, we define an \textbf{Optimal Encoder} $\mathrm{Enc}_0$. This agent acts as a path-finding function on the Stern-Brocot tree $\operatorname{SB}$, mapping indistinguishable microstates to their most parsimonious representations.

The encoding process follows the conjugate symmetry of the symplectic phase space $\Gamma$:

$$
\mathrm{Enc}_{0,q}(q_?, \mathcal{P}) \mapsto \mathbf{s_{0,q}}, \mathbf{Q_0}
$$

\begin{align}
\mathrm{Enc}_{0,q}(q_?, \mathcal{P}) &\mapsto \langle \mathbf{s_{0,q}} \cong \mathbf{Q_0} \rangle \\
\mathrm{Enc}_{0,p}(p_?, \mathcal{Q}) &\mapsto \langle \mathbf{s_{0,p}} \cong \mathbf{P_0} \rangle
\end{align}

\noindent where:
\begin{itemize}
    \item $\mathrm{Enc}_0$: The \textbf{Optimal Agent} or mapping function (AIT: $E^*$).
    \item $q_?, p_?$: The raw \textbf{microstates} within the symplectic quantum of action $A_{d\Gamma}$.
    \item $\mathcal{P}, \mathcal{Q}$: The \textbf{boundary constraints} acting as auxiliary information or priors.
    \item $\mathbf{s_{0,q}}, \mathbf{s_{0,p}}$: The \textbf{minimal binary programs} (bitstrings) of length $K(\mathbf{Q_0})$ and $K(\mathbf{P_0})$.
    \item $\mathbf{Q_0}, \mathbf{P_0}$: The \textbf{macrostates} resulting from the action of $\mathcal{U}_L$ on the programs.
    \item $\cong$: The \textbf{isomorphism} between the algorithmic description (program) and the physical state (geometry).
\end{itemize}

Consistent with the \textit{Source Coding Theorem} and the work of Li and Vitányi, the encoder $\mathrm{Enc}_0$ is defined by its ability to reach the \textbf{Kolmogorov limit} where the program length $|\mathbf{s_0}|$ equals the algorithmic complexity of the macrostate. The subsequent Landauer erasure of $\mathbf{s_0}$ by the machine $\mathcal{U}_L$ accounts for the thermodynamic cost associated with the collapse of the microstate set $\{q_?\}$ to the singular representation $\mathbf{Q_0}$.


\begin{align}
\mathrm{Enc}_{0,q}(q_?, \mathcal{P}) &\mapsto \langle \mathbf{s_{0,q}} \cong \mathbf{Q_0} \rangle \\
\mathrm{Enc}_{0,p}(p_?, \mathcal{Q}) &\mapsto \langle \mathbf{s_{0,p}} \cong \mathbf{P_0} \rangle
\end{align}

\section{Algorithmic Uncertainty and Joint Complexity}

The relationship between the conjugate programs $\mathbf{s_{0,q}}$ and $\mathbf{s_{0,p}}$ is governed by the \textbf{algorithmic distance} $D$ and the total information content required to describe the symplectic state.

\subsection*{1. Algorithmic Distance}
The distance between the position and momentum programs is defined by the length of the shortest program that transforms one into the other:
\begin{equation}
    D(\mathbf{s_{0,q}}, \mathbf{s_{0,p}}) = \max\{ K(\mathbf{s_{0,q}} | \mathbf{s_{0,p}}), K(\mathbf{s_{0,p}} | \mathbf{s_{0,q}}) \}
\end{equation}
On the \textbf{Stern-Brocot tree} ($\operatorname{SB}$), this corresponds to the number of tree-traversal steps (branch shifts) required to move from node $\mathbf{Q_0}$ to node $\mathbf{P_0}$.

\subsection*{2. The Joint Complexity Bound}
By the \textbf{Invariance Theorem}, the joint description of the conjugate pair is bounded by the log-volume of the symplectic quantum of action $A_{d\Gamma}$:
\begin{equation}
    K(\mathbf{s_{0,q}}, \mathbf{s_{0,p}}) \ge \log_2 \left( \frac{A_{d\Gamma}}{\pi \hbar} \right) + C
\end{equation}
This implies that the total bit-depth of the encoder $\mathrm{Enc}_0$ is constrained; increasing the precision (length) of $\mathbf{s_{0,q}}$ necessarily limits the available information for $\mathbf{s_{0,p}}$ to remain within the same symplectic blob.

\subsection*{3. Symplectic Mutual Information}
The information shared between the conjugate bases is given by:
\begin{equation}
    I(\mathbf{s_{0,q}} : \mathbf{s_{0,p}}) = K(\mathbf{s_{0,q}}) + K(\mathbf{s_{0,p}}) - K(\mathbf{s_{0,q}}, \mathbf{s_{0,p}})
\end{equation}
This \textbf{Mutual Information} quantifies the degree of redundancy provided by the boundary conditions $\mathcal{P}$ and $\mathcal{Q}$.

\subsection*{Summary for Implementation}
\begin{itemize}
    \item \textbf{Isomorphism:} Distance in bit-space (AIT) $\cong$ distance in phase-space (Geometry).
    \item \textbf{SB Traversal:} Moving between conjugates is a \textbf{path-switching} operation on the tree.
    \item \textbf{Landauer Cost:} The total action $A_L$ must account for the erasure of the \textbf{joint complexity} $K(\mathbf{s_{0,q}}, \mathbf{s_{0,p}})$.
\end{itemize}

\section{de Gosson}

The characteristic action (facilitating dimensionless, normalized units) is defined as 
\begin{equation}
    A_0 = q_0 p_0 = \hbar.
\end{equation}

The de Gosson quantum of action, representing the minimum symplectic capacity of a phase-space ``blob,'' is defined by the area of an ellipse with conjugate diameters $\Delta q$ and $\Delta p$:
\begin{equation}
    A_{d\Gamma} = \frac{\pi}{4} \Delta q \Delta p = \pi \hbar.
\end{equation}

For a system with macroscopic semi-axes $\mathcal{Q}$ and $\mathcal{P}$ (where $\Delta Q = 2q_0\mathcal{Q}$ and $\Delta P = 2p_0\mathcal{P}$), the total system action $A_\Gamma$ is the sum of these elementary quanta:
\begin{equation}
    A_\Gamma = \sum A_{d\Gamma} = \frac{\pi}{4} (2 q_0 \mathcal{Q}) (2 p_0 \mathcal{P}) = \pi \hbar \mathcal{Q} \mathcal{P} \ge \pi \hbar.
\end{equation}

The counting logic dictates that the system contains a discrete number of action quanta:
\begin{equation}
    N_\Gamma = \mathcal{Q}\mathcal{P}.
\end{equation}
In this framework, a macroscopic system is characterized by $N_\Gamma \gg 1$, while the quantum ground state is recovered at the lower bound, $N_\Gamma = 1$.


\section{Landauer}

Landauer erasure over a characteristic time $\tau_L$ requires an action $A_L$ bounded by the symplectic quantum of action,

\begin{equation}
k_{B}T\ln(2)K(\mathbf{Q_0}, \mathbf{P_0})\tau_L \le A_L \le A_{d \Gamma} = \pi \hbar,
\end{equation}

and the limited action capacity of one quantum of action $A_{d\Gamma}$ requires the bifurcation of the action into a conjugate action pair,

\begin{equation}
|q_{?} - \mathbf{Q_0}| \le \mathcal{P}^{-1}.
\end{equation}

and

\begin{equation}
|p_{?} - \mathbf{P_0}| \le \mathcal{Q}^{-1},
\end{equation}

where $K(\mathbf{Q_0}, \mathbf{P_0}) = |\mathbf{s_{Q_0}}| + |s_{P_0}|$.


The Landauer erasure time $\tau _{L}$ for the ground state $K(\mathbf{Q}_{\mathbf{0}})=1$ is 

\begin{equation}
\tau _{L}\le \frac{\pi \hbar }{k_{B}T\ln (2)}=\frac{\pi }{\ln (2)}\tau _{\hbar },
\end{equation}

where $\tau _{\hbar }=\frac{\hbar }{k_{B}T}$ is the Planckian dissipation time. Here, the factor of $\pi $ accounts for the elliptical geometry of the quantum of action, while $\ln (2)$ represents the entropic cost of a single bit erasure. This establishes that the erasure time is a specific thermodynamic-geometric scaling of the absolute quantum-thermal speed limit.

\section{The Thermodynamic Necessity of Conjugate Action Quanta}

\subsection{The Entropic Conflict}
The core equation sets a strict cap on how much thermodynamic complexity a single "action blob" can contain:
\[k_{B}T\ln(2)K(\mathbf{Q_0}, \mathbf{P_0})\tau_L \le A_L \le \pi \hbar\]
If one attempts to place all information for both position and momentum into one blob, they run into the fundamental limit imposed by the Planck constant $\hbar$. A single quantum blob in phase space represents an absolute minimum volume, leaving no room for precise conjugate values.

\subsection{The Additivity of Action-Information and the Conjugate Leap}

By defining the joint complexity of the system as the sum of independent binary programs on the universal Landauer machine $\mathcal{U}_{\mathrm{L}}$, we establish that $K(\mathbf{Q_0}, \mathbf{P_0}) = |\mathbf{s_{Q_0}}| + |\mathbf{s_{P_0}}|$. Consequently, the Landauer bound on the characteristic erasure time $\tau_L$ within the symplectic quantum of action $A_L \le \pi \hbar$ is expressed as:

\begin{equation}
k_{B}T\ln(2) \left( |\mathbf{s_{Q_0}}| + |\mathbf{s_{P_0}}| \right) \tau_L \le \pi \hbar.
\end{equation}

This identity reveals that the "leap" from a single microstate continuum to a conjugate pair of observables is an thermodynamic necessity. Because the information budget for a discrete reality is limited by the \textbf{symplectic capacity} \cite{deGosson2011} of the ``quantum blob'' ($\pi \hbar$), the system cannot resolve both position and momentum within a single geometric orientation. 

To maintain the total action while reducing complexity through \textbf{Landauer erasure} \cite{Landauer1961}, the universe bifurcates the action into a \textbf{conjugate pair}:
\begin{itemize}
    \item \textbf{The Q-Blob:} Minimizes position uncertainty by utilizing the thermodynamic ``slack'' of the conjugate momentum program.
    \item \textbf{The P-Blob:} Minimizes momentum uncertainty by utilizing the thermodynamic ``slack'' of the conjugate position program.
\end{itemize}

This duality effectively maps the bulk thermodynamic information into the discrete, erasable bits of our observable world, ensuring that the erasure time adheres to the absolute \textbf{Planckian dissipation limit} $\tau_{\hbar} = \frac{\hbar}{k_B T}$, scaled by the factor $\frac{\pi}{\ln(2)}$ which bridges de Gosson's elliptical geometry with thermodynamic entropy.

\subsection{The thermodynamic-Geometric Solution}
To maintain a consistent physical description (which allows both position-space and momentum-space realities), the system must utilize two linked quanta of action---a \textbf{conjugate pair}.
\begin{itemize}
    \item \textbf{The "Q Blob" (Position Focus):} This state prioritizes the resolution of position information. To achieve a collapse to a discrete observable $\mathbf{Q_0}$, the system utilizes the thermodynamic "slack" of the conjugate momentum, satisfying:
    \begin{equation}
        |q_{?} - \mathbf{Q_0}| \le \mathcal{P}^{-1}
    \end{equation}
    \begin{equation}
        \delta_q = q_{?} - \mathbf{Q_0}
    \end{equation}
    \begin{equation}
        q = \frac{\delta_q}{p_0} \mathcal{A}
    \end{equation}
    Geometrically, this corresponds to a horizontal orientation in phase space where position uncertainty is minimized relative to the available action $\mathcal{A}$.

    \item \textbf{The "P Blob" (Momentum Focus):} This state prioritizes the resolution of momentum information. To achieve a collapse to a discrete observable $\mathbf{P_0}$, the system utilizes the thermodynamic "slack" of the conjugate position, satisfying:
    \begin{equation}
        |p_{?} - \mathbf{P_0}| \le \mathcal{Q}^{-1}
    \end{equation}
    \begin{equation}
        \delta_p = p_{?} - \mathbf{P_0}
    \end{equation}
    \begin{equation}
        p = \frac{\delta_p}{q_0} \mathcal{A}
    \end{equation}
    Geometrically, this produces the vertically elongated structure seen in the numerical analysis of the oscillator.
\end{itemize}


The series of plots illustrates this: as the "transition" progresses (representing energy levels of a Quantum Harmonic Oscillator), the two blobs morph while maintaining their duality, confirming their linked nature as an thermodynamic necessity.

\subsection{The Origin of Quantum Fluctuations}

We propose that the \textbf{quantum fluctuation}, $\delta_q$, is the physical isomorphism of thermodynamic information loss. While traditionally viewed as stochastic noise, $\delta_q$ is revealed here to be the physical coordinate displacement resulting from the erasure of the microstate program $s_?$ into the minimal macrostate program $\mathbf{s_0}$. 

On the thermodynamic side, the universe discards a string length $\Delta I = |s_?| - |\mathbf{s_0}|$. On the physical side, this is expressed as the fluctuation:
\begin{equation}
    \delta_q = q_{?} - \mathbf{Q_0}
\end{equation}
This mapping identifies the Heisenberg Uncertainty not as a limit of nature's "knowledge," but as the manifestation of the Landuaer eraser projected into the physical world.

% --- Conceptual Translation: Detroit speak to Physics Formalism ---

\begin{quotation}
\noindent \textbf{Conceptual Framework:} 
Our computational simulations reveal state distributions characterized by node counts that coincide with the energy bands predicted by canonical wave mechanics; however, these states are neither statistically favored nor exclusive within the model. Instead, they emerge as a subset of manifold configurations accessible under the specific phase-space volume dictated by the system’s action.

\medskip

In this framework, analyzing our numerical output is functionally equivalent to the inferential processes of quantum state tomography. Crucially, our results are derived solely from classical statistical mechanics---utilizing a Maxwell--Boltzmann--Planck thermodynamic foundation---precluding the need for non-local entanglement or wavefunction collapse. Consequently, we propose that the physical vacuum provides no privileged status to Schr\"{o}dinger-eigenstate predictions, which instead appear as specific manifestations within a broader thermodynamic distribution of action-constrained microstates.
\end{quotation}

\section{Notation}
The variables used primarily throughout this manuscript are dimensionless values—such as coordinate $\mathcal{L}$, momentum $\mathcal{P}$, energy $\mathcal{E}$, and time $\mathcal{T}$—normalized to their characteristic base units ($q_0, p_0, E_0, t_0$). We define the fundamental \textit{quantum of action} as $\pi \hbar = \frac{\pi}{4} q_0 p_0 = \pi \hbar$, which establishes the phase-space resolution of the system. While the base units ($q_0, p_0$) define the system's characteristic scales, the dynamic state resolutions ($\Delta q, \Delta p$) are captured by the dimensionless parameters $\mathcal{L}$ and $\mathcal{P}$.

Physical quantities are normalized as follows:

\noindent

\textbf{REVISIT: Normalized Notation} \\
Microstate: $\mu_q =  q_?\Delta q, \quad \mu_p =  p_?\Delta p $ \\
Macrostate: $q = \mathbf{Q_0} \Delta q, \quad p = \mathbf{P_0} \Delta p$ \\
Coordinate: $q(t) = \Delta q \mathcal{L}(t)$ \\
Momentum: $p(t) = \Delta p \mathcal{P}(t)$ \\
Time: $t = t_0 \mathcal{T}$ \\
Energy: $E = E_0 \mathcal{E}$ \\
Quantum of Action: $\pi \hbar = \frac{\pi}{4} \Delta q \Delta p$ \\
Action: $A = \pi \hbar \mathcal{A}$ \\
Characteristic Action: $\frac{\pi}{4} \Delta q \Delta p \coloneq \pi \hbar$ \\
Normalized Action: $\mathcal{A} = \mathcal{P}^2 + \mathcal{L}^2$ \\
Hamiltonian: $\mathcal{H} = E_0 \mathcal{A} = \pi \hbar (\mathcal{P}^2 + \mathcal{L}^2) f$ \\
Landauer Machine: $\mathcal{U}_{\mathrm{L}}$ \\
Binary Program: $s$ \\
Minimal Binary Program: $\mathbf{s_0}$ \\
Binary Program Length: $K(\mathbf{Q_0}) = |\mathbf{s_0}|$ \\
Erasure Mapping: $\mathcal{U}_{\mathrm{L}} : \{q_?\} \mapsto \mathbf{Q_0}$ \\
Halt Criteria: $|q_? - \mathbf{Q_0}| < \mathcal{P}^{-1}$

\textbf{Normalized Notation} \\
Microstate: $\mu_q = q_0 q_?, \quad \mu_p = p_0 p_?$ \\
Macrostate: $q = \mathbf{Q_0} q_0, \quad p = \mathbf{P_0} p_0$ \\
Coordinate: $q(t) = q_0 \mathcal{L}(t)$ \\
Momentum: $p(t) = p_0 \mathcal{P}(t)$ \\
Time: $t = t_0 \mathcal{T}$ \\
Energy: $E = E_0 \mathcal{E}$ \\
Quantum of Action: $\pi \hbar = \frac{\pi}{4} \Delta q \Delta p$ \\
Action: $A = \pi \hbar \mathcal{A}$ \\
Characteristic Action: $\frac{\pi}{4} q_0 p_0 \coloneq \pi \hbar$ \\
Normalized Action: $\mathcal{A} = \mathcal{P}^2 + \mathcal{L}^2$ \\
Hamiltonian: $\mathcal{H} = E_0 \mathcal{A} = \pi \hbar (\mathcal{P}^2 + \mathcal{L}^2) f$ \\
Landauer Machine: $\mathcal{U}_{\mathrm{L}}$ \\
Binary Program: $s$ \\
Minimal Binary Program: $\mathbf{s_0}$ \\
Binary Program Length: $K(\mathbf{Q_0}) = |\mathbf{s_0}|$ \\
Erasure Mapping: $\mathcal{U}_{\mathrm{L}} : \{q_?\} \mapsto \mathbf{Q_0}$ \\
Halt Criteria: $|q_? - \mathbf{Q_0}| < \mathcal{P}^{-1}$


\subsection{REVIST: The Canonical Flow}

The resolution of a physical state is determined either by the probabilistic spread of its conjugate observables, $\sigma_{q}\sigma_{p} \ge \hbar/2$, or by its action boundary $A$ in phase space:

$$A = \frac{\pi}{4} \Delta p \Delta q \ge \pi \hbar.$$

Within this hard action boundary, the entropy of the state is physically constrained by the minimum symplectic volume $A = \pi \hbar$. Because all microstates $\{\mu_q\} = \{q_{?} \Delta q\}$ within this quantum of action must be operationally indistinguishable ($\delta_{q} \le \Delta q \implies \frac{\pi}{4} \delta_{q} \Delta p \le \pi \hbar$), it is a thermodynamic necessity that only the macrostate $\mathbf{Q_0}\Delta q$ which maximizes system entropy by minimizing local complexity is physically possible: $\delta_{q} = |q_{?} - \mathbf{Q_0}|\Delta q $, where $q_{?}$ and $\mathbf{Q_0}$ are normalized and dimensionless. Defining the scale as $\Delta p = p_{0}\mathcal{P}$, we have:

$$\left(\frac{\pi}{4} q_{0} p_{0}\right) |q_{?} - \mathbf{Q_0}| \mathcal{P} \le \pi \hbar.$$

Recognizing the characteristic action of the system is defined by $\frac{\pi}{4} q_{0} p_{0} \coloneq \pi \hbar$, we arrive at:

\begin{equation}
|q_{?} - \mathbf{Q_0}| \le \mathcal{P}^{-1}.
\end{equation}

This thermodynamic collapse is executed by a universal computing machine we call the Landauer eraser:

\begin{equation}
\mathcal{U}_{\mathrm{L}}(\{q_{?}\}) \mapsto \mathbf{Q_0}
\end{equation}

which irreversibly erases the thermodynamic information of the indistinguishable microstates $\{q_{?}\} : \pi \hbar$, mapping them to the one minimal complexity state $K(\mathbf{Q_0}) = |\mathbf{s_0}|$ defined by the minimal program $\mathbf{s_0}$, where $K(\mathbf{Q_0})$ is the Kolmogorov complexity and $|\mathbf{s_0}|$ is the length of the program's binary string.

The bound $|q_{?} - \mathbf{Q_0}| \le \mathcal{P}^{-1}$ establishes a fundamental thermodynamic event horizon for the physical state. Within this framework, the normalized momentum $\mathcal{P}$ functions as a computational budget; higher momentum scales provide the necessary action to sustain the thermodynamic information required for higher-resolution microstates. Conversely, when the work required to maintain the distinction between a microstate $q_{?}$ and its macrostate $\mathbf{Q_0}$ exceeds the available action, the Landauer eraser enforces a transition to the state of bounded thermodynamic complexity. This thermodynamic collapse effectively truncates the bit-string representation of the state, mapping the indistinguishable sub-resolution information to the minimal complexity program $\mathbf{s_0}$ consistent with the $\pi \hbar$ boundary.

\subsection{The Canonical Flow}

The resolution of a physical state is determined either by the probabilistic spread of its conjugate observables, $\sigma_{q}\sigma_{p} \ge \hbar/2$, or by its action boundary $A$ in phase space:

$$A = \frac{\pi}{4} \Delta p \Delta q \ge \pi \hbar.$$

Within this hard action boundary, the entropy of the state is physically constrained by the minimum symplectic volume $A = \pi \hbar$. Because all microstates $\{\mu_q\} = \{q_{0}q_{?}\}$ within this quantum of action must be operationally indistinguishable ($\delta_{q} \le \Delta q \implies \frac{\pi}{4} \delta_{q} \Delta p \le \pi \hbar$), it is a thermodynamic necessity that only the macrostate $q_{0}\mathbf{Q_0}$ which maximizes system entropy by minimizing local complexity is physically possible: $\delta_{q} = q_{0}|q_{?} - \mathbf{Q_0}|$, where $q_{?}$ and $\mathbf{Q_0}$ are normalized and dimensionless. Defining the scale as $\Delta p = p_{0}\mathcal{P}$, we have:

$$\left(\frac{\pi}{4} q_{0} p_{0}\right) |q_{?} - \mathbf{Q_0}| \mathcal{P} \le \pi \hbar.$$

Recognizing the characteristic action of the system is defined by $\frac{\pi}{4} q_{0} p_{0} \coloneq \pi \hbar$, we arrive at:

\begin{equation}
|q_{?} - \mathbf{Q_0}| \le \mathcal{P}^{-1}.
\end{equation}

This thermodynamic collapse is executed by a universal computing machine we call the Landauer eraser:

\begin{equation}
\mathcal{U}_{\mathrm{L}}(\{q_{?}\}) \mapsto \mathbf{Q_0}
\end{equation}

which irreversibly erases the thermodynamic information of the indistinguishable microstates $\{q_{?}\} : \pi \hbar$, mapping them to the one minimal complexity state $K(\mathbf{Q_0}) = |\mathbf{s_0}|$ defined by the minimal program $\mathbf{s_0}$, where $K(\mathbf{Q_0})$ is the Kolmogorov complexity and $|\mathbf{s_0}|$ is the length of the program's binary string.

The bound $|q_{?} - \mathbf{Q_0}| \le \mathcal{P}^{-1}$ establishes a fundamental thermodynamic event horizon for the physical state. Within this framework, the normalized momentum $\mathcal{P}$ functions as a computational budget; higher momentum scales provide the necessary action to sustain the thermodynamic information required for higher-resolution microstates. Conversely, when the work required to maintain the distinction between a microstate $q_{?}$ and its macrostate $\mathbf{Q_0}$ exceeds the available action, the Landauer eraser enforces a transition to the state of bounded thermodynamic complexity. This thermodynamic collapse effectively truncates the bit-string representation of the state, mapping the indistinguishable sub-resolution information to the minimal complexity program $\mathbf{s_0}$ consistent with the $\pi \hbar$ boundary.

\subsection*{Informational Substrate and the Universal Computer}

In defining the bit-length $|s^*|$ of a physical state, we must establish the computational substrate upon which these specifications are constructed. Consistent with Zurek's observation that the selection of a specific universal computer $\mathcal{U}$ is mathematically arbitrary for the determination of thermodynamic complexity, we adopt a framework scaled to the fundamental units of phase space.

\subsection*{The Stern-Brocot Computer}
Throughout this paper, we utilize a universal computer $\mathcal{U}$ based on the \textbf{Stern-Brocot construction}, with its output scaled to the characteristic Planck-scale unit $x_0$:
\begin{equation}
\mathcal{U} := x_0 \operatorname{SB}
\end{equation}
By defining the computer in this manner, we establish a direct isomorphism between the physical resolution limit $1/\mathcal{P}$ and the thermodynamic depth of the coordinate specification. The Hamiltonian's action budget $S$ thus dictates the maximum depth of the Stern-Brocot path, ensuring that the bit-length $|s^*|$ remains physically bounded by the symplectic floor.

\subsubsection*{The Invariance of the Computational Substrate}
While the specific bit-length of a coordinate is dependent on the selection of the universal computer $\mathcal{U}$, the validity of the measure rests entirely on the consistency of the chosen substrate. As established by Li and Vit\'{a}nyi \cite{li_vitanyi_2026} and emphasized by Zurek \cite{zurek_1989}, any comparative analysis of bit-strings $s$ remains objective so long as all strings are executed on the same reference machine. 

This consistency addresses the objection that highly symmetric irrational values, such as the golden ratio ($\phi$), could be "compressed" into a trivial program (e.g., "repeat 10"). While one is free to imagine a symbolic computer that outputs $\frac{1+\sqrt{5}}{2}$ as a single primitive—a device whose name was Euclid—such a computer represents a different physical mapping. Within the fixed framework of $\mathcal{U}_{SB}$, every bit-string $s$ is parsed strictly as a path to a rational terminus $X \in \mathbb{Q}$ that saturates the Heisenberg cell. Because we maintain $\mathcal{U}_{SB}$ as our invariant reference, the bit-length $|s^*|$ provides a consistent and objective measure of the Hamiltonian action required for state specification.


\textit{Note: For a formal treatment of the Stern-Brocot tree as an thermodynamic complexity generator and its mapping to rational coordinate approximations, see Appendix A.}

\section*{Symplectic Reciprocity and Conjugate Specification}

While the derivations in this paper primarily focus on the specification of the position coordinate $x$ relative to the momentum scaling factor $\mathcal{P}$, the theory maintains strict \textbf{Symplectic Reciprocity}. 

\subsection*{The Symmetry of Bit-Allocation}
In a symplectic phase space, the informational budget provided by the Hamiltonian $H$ is a shared resource between conjugate pairs. The resolution of position is "paid for" by the action capacity of momentum, and vice versa:
\begin{equation}
|s_x^*| \propto \mathcal{P} \quad \text{and} \quad |s_p^*| \propto \mathcal{L}
\end{equation}
To avoid redundant formalism, we define the \textbf{Conjugate Bound}: any thermodynamic result derived for the specification of $x$ using the computer $\mathcal{U} := x_0 \operatorname{SB}$ holds isomorphically for the specification of $p$ using the conjugate computer $\mathcal{U}^\dagger := p_0 \operatorname{SB}$.

\subsection*{The Total Informational Volume}
The fundamental limit of the theory is not on a single coordinate, but on the \textbf{Total Specification}:
\begin{equation}
|s_x^*| + |s_p^*| \leq \frac{S}{S_0}
\end{equation}
This ensures that any "gain" in bit-length for the position coordinate necessitates a corresponding "erasure" or lack of specification in the momentum coordinate, preserving the hard symplectic boundary $h/2$. Throughout the following sections, coordinate-specific results should be understood as projections of this unified 2D informational constraint.


\subsection*{Operationally Indistinguishable Microstates}

We define $x_{?}$ as an unknowable value, a pre-erasure microstate: the original, fine-grained coordinate of the system. We say that information has been erased during an observation because we will never find out what the state $x_{?}$ was originally. Inside the quantum of action $S = x_0 p_0 = h/2$ (the coarse-grained minimal Heisenberg cell), $x_{?}$ is operationally indistinguishable from the resulting observed state $x$ such that 
$$\frac{\pi}{4}\mathcal{L}|q_{?}-\mathbf{Q_0}|\Delta p<h/2.$$ 

Because the specific value of $x_{?}$ is rendered forever unknowable by this many-to-one compression, the erasure is logically irreversible. 

 The difference between $x$ and $x_{?}$ is that the thermodynamic complexity $|b_?|$ of $x_?$ is greater than the thermodynamic complexity of the minimal thermodynamic complexity value $|b_?| > |b^*|$ within the quantum of action.

\medskip

We define $b$ as the program bit string of length $|b|$ required to specify 
the observed state $x$. Because every additional bit in the string incurs 
a thermodynamic erasure cost of $kT \ln 2$, the physical system is subject 
to an thermodynamic-thermodynamic pressure. It is therefore a thermodynamic 
necessity that from the ensemble of available microstates 
$\{x_? : |x_? - x| \Delta p < h/2 \}$, the observed value $x$ must 
correspond to the unique bit string $\mathbf{b^*}$ of minimal length 
$|\mathbf{b^*}|$. This is expressed as the thermodynamic mapping 
$$x = \mathcal{U}(\mathbf{b^*})$$ 
where $\mathcal{U}$ represents the universal physical constructor that 
identifies the state $x$ as the output of the minimal program $\mathbf{b^*}$ 
that satisfies the resolution limit $|x_{?} - x| \Delta p < h/2$. In this 
framework, the transition to a quantized state is a thermodynamic selection of the 
lowest-complexity representation permitted by the system's energy scale. 
This selection is governed by the principle that the universe naturally 
settles into the state of \textit{minimal informational action} [1]. As 
Landauer established, longer strings represent higher-potential energy states 
for the coupled system-reservoir ensemble, and thus $\mathbf{b^*}$ is the 
only state whose entropy can be fully dissipated by the local thermal 
background [2]. Furthermore, in the framework of the 
\textit{Foundations of thermodynamic Thermodynamics} [3], $\mathbf{b^*}$ 
represents the "coarsest" thermodynamic descriptor that remains invariant 
under environmental interaction. Any higher-complexity representation $b_{n}$ 
is thermodynamically unstable; the extra information is effectively 
"scrambled" by the thermal reservoir, forcing the system to halt at the 
minimal bit string $\mathbf{b^*}$. It is this thermodynamic necessity of 
erasing the original state and selecting the shortest description 
$\mathbf{b^*}$ that forces the phase-space to emerge as a quantized 
structure rather than a continuous one.

\medskip

The implementation of the universal constructor $\mathcal{U}$ is operationally realized through the Stern-Brocot decoding function $\operatorname{SB}$, which maps the minimal program $\mathbf{b^*}$ to a unique rational $X \in \mathbb{Q}$. The constructor is thus defined by scaling this rational address with the fundamental length unit:
\begin{equation}
\mathcal{U}(\mathbf{s_0}) = x_0 \operatorname{SB}(\mathbf{s_0})
\end{equation}
In this framework, the Stern-Brocot tree—a complete binary search tree of all positive rationals—acts as the decoding architecture for phase-space. The minimal program $\mathbf{b^*}$ is an ordered bit string where each bit represents a discrete branching instruction: a '0' for a left (L) traversal and a '1' for a right (R) traversal. The constructor $\mathcal{U}$ reads these instructions to navigate from the root toward the rational $X$ that identifies the quantized state.
\begin{equation}
x = x_0 \operatorname{SB}(\mathbf{b^*}) = x_0 X
\end{equation}
This traversal is not indefinite; it is governed by the thermodynamic necessity to halt. As the tree is descended, the bit-length $|\mathbf{b^*}|$ increases, incurring an cumulative erasure cost of $|\mathbf{b^*}|kT\ln 2$. Consequently, the thermodynamic thermodynamics select the exact level where the resolution limit $|x_{?} - x_0 \operatorname{SB}(\mathbf{b^*})| \Delta p < h/2$ is first satisfied. By embedding the Stern-Brocot tree directly into the constructor, the physical system ensures that the resulting quantized value $x$ is the most thermodynamically efficient representation of the  available microstates 
$\{x_? : |x_? - x_0 X| \Delta p < h/2 \}$. This mechanism minimizes the complexity of the physical record by selecting the most accessible rational convergent within the resolution bound, effectively forcing the phase-space to emerge as a quantized structure from the thermodynamic limit of information erasure.

\medskip
The physical impossibility of resolving irrational states is best illustrated by the Golden Ratio, 
$$\phi \approx 1.618 \ 033 \ 988 \ 749 \ \dots$$ 
In the Stern-Brocot implementation of $\mathcal{U}$, the program required to compute $x = \phi x_0$ is a non-terminating instruction set $b_{\phi} = (1, 0, 1, 0, \dots)$. This infinite string exposes a critical divergence between the Shannon entropy $H_{Sh}$---the \textit{ensemble average} of information---and the thermodynamic entropy $H_{alg}$, which quantifies the physical complexity of the \textit{individual microstate}. 

Following the framework of thermodynamic thermodynamics \cite{Ebtekar2025}, we define the relationship between the ensemble average and the individual microstate complexity:
\begin{equation}
    H_{Sh} = -\sum_{i \in \{0,1\}} w_i \log_2 w_i, \quad H_{alg}(x_{?}) = |b|
\end{equation}
where $w_i$ represents the statistical probability of a branching instruction $i$ within the ensemble, and $H_{alg}(x_{?})$ is the actual length of the instruction set $b$ required to manifest the specific microstate $x_?$. While the Shannon entropy of the alternating rule for $\phi$ remains finite and low, the physical Landauer cost of manifesting the individual microstate is driven by the absolute bit-length of its program:
\begin{equation}
    \mathcal{W} = \lim_{|b| \to \infty} |b| k T \ln 2 = \infty
\end{equation}

This distinction is fundamental: whereas Shannon entropy measures our mean ignorance of an ensemble, the thermodynamic complexity $|b|$ quantifies the actual thermodynamic weight of the specific microstate at each step of its resolution. As Zurek established in his formulation of \textit{physical entropy} \cite{Zurek1989}, the total entropy $S_d = H_{alg} + H_{Sh}$ must account for the information already manifested in the record. Because every additional bit in $b$ incurs a cumulative energy cost, the physical system is subject to an informational pressure that favors the state of minimal complexity. Since no local thermal reservoir possesses the infinite energy capacity required to finalize the non-terminating program $b_\phi$, the constructor $\mathcal{U}$ is forced to \textit{halt} at the minimal program $\mathbf{b^*}$---the unique rational convergent that satisfies the \textbf{Heisenberg resolution limit} $|x_{?} - x| \Delta p < h/2$ with the least thermodynamic work. This thermodynamic barrier proves that irrational coordinates are physically unrealizable, leaving the quantized rational record as the only possible state of the phase-space.


\subsection{Statistical Heisenberg Uncertainty Principle}

$$
\sigma_x \sigma_p \ge \frac{\hbar}{2}
$$

\subsection{Symplectic Heisenberg Uncertainty Principle}

\subsubsection{The Quantum of Action}

$$
S_0 = x_0 p_0 = \pi \hbar
$$

\subsubsection{The Symplectic Capacity}

$$
S = S_0 ( \mathcal{L} \times \mathcal{P}) \ge \pi \hbar
$$

$$
S = c(\Omega)
$$

where $L$ and $P$ are dimensionless scaling factors of the characteristic position $x_0$ and momentum $p_0$.

$$
\Delta x \Delta p \ge \pi \hbar
$$

$$
\Delta x := \mathcal{L} x_0 \quad \Delta p := \mathcal{P} p_0
$$

\subsection{Quantum Harmonic Oscillator}

$$
H = \frac{\mathcal{P}^2 p_0^2}{2m} + \frac{1}{2}m\omega^2 \mathcal{L}^2 x_0^2
$$

$$
H = \pi \hbar(\mathcal{P}^2 + \mathcal{L}^2)f
$$

\subsection{Fundamental Fraction}


\begin{quote}

We say that information has been erased during the compression because we will never find out where the molecule was originally.

\hfill --- \textit{The Physics of Forgetting}

\end{quote}

We say that information has been erased $|\mathbf{s_0}|kT\ln{2}$ when the microstate collapses to the macrostate $q_? \rightarrow \mathbf{Q_0}, \ \mu_q = q_?q_0, \ q =  \mathbf{Q_0} q_0, \ \mathbf{Q_0} \in \mathbb{Q}$ (dimensionless, normalized units) because we will never find out what the state $q_?$ was originally. State values within the coarse-grained Heisenberg uncertainty cell (quantum blob) are operationally indistinguishable 
$\frac{\pi}{4}q_0\mathcal{L}|q_? - \mathbf{Q_0}| \Delta p < h/2$. Each $\mathbf{Q_0}$ corresponds to a bit string $\mathbf{s_0}$ which is unique for the given physical system. For any physical system $\Delta q := q_0\mathcal{L}, \ \Delta p := p_0 \mathcal{P}$ is defined by the Hamiltonian. The dimensionless normalized factor $\mathcal{P}$ indicates the momentum and also the resolution limit of the conjugate observable and is determined by the Hamiltonian's energy scale. We find $\mathbf{Q_0}$ and $\mathbf{s_0}$ using the Stern-Brocot tree such that $|q_?/q_0 - \mathbf{Q_0}| < 1 / \mathcal{P} \implies \delta_q \Delta p < h/2$, using the hard symplectic boundary $h/2$ of the Heisenberg cell as opposed to the fuzzy statistical one $\hbar/2$. The spread of the position and momentum as $\mathcal{P}$ increases the ensemble Shannon entropy $\mathrm{H}$ while the increasing resolution $1/\mathcal{P}$ increases the thermodynamic entropy $\mathrm{K}$ required to erase the information about the unknown state.


\subsection{On Sub-Planck Scales}
We suggest that the sub-Planck structures identified by Zurek are not violations of the fundamental quantum of action, but rather a manifestation of conjugate quanta of action. By allowing the two quantum blobs $A=\pi \hbar$ to evolve into an elliptical geometry defined by our scaling factors $\mathcal{Q}$ and $\mathcal{P}$, one can recover Zurek’s high-resolution interference features. In this view, the sub-Planck scale actually represents a one-dimensional resolution limit $\delta_{q}$ or $\delta_{p}$ of a squeezed symplectic invariant, rather than a reduction of the total phase-space area below the de Gosson boundary.

\section*{Informational Action Bound Theory}

\subsection*{Context: The Complexity-Action (CA) Conjecture}

The proposal that physical specification is bounded by action finds significant precedent in the \textbf{Complexity-Action (CA) Conjecture}, a major hypothesis in high-energy physics and AdS/CFT duality proposed by Susskind, Brown, and colleagues.

\subsection*{The Central Hypothesis}
The CA conjecture posits that the quantum complexity ($\mathcal{C}$) of a boundary state is exactly proportional to the classical action ($S$ or $\mathcal{I}$) of a specific spacetime region known as the Wheeler-DeWitt patch. This is formally expressed as:
\begin{equation}
\mathcal{C} = \frac{\mathcal{I}}{\pi \hbar}
\end{equation}

\subsection*{Relevance to the Coordinate-Action Bound}
The CA conjecture provides a direct academic foundation for the claim that "one cannot specify the value of any physical state with a bit length greater than the capacity of the action." It implies that action is not merely a dynamical integral, but the physical manifestation of thermodynamic complexity. 

While the CA conjecture typically applies to holographic boundary states, the \textbf{Coordinate-Action Bound} extends this logic to the minimal bit-length specification of individual phase-space coordinates ($x, p$), identifying the symplectic quantum $h$ as the fundamental resolution limit for such physical information.

\section*{The Minimal Action-Complexity Bound}

This section summarizes the fundamental limit of physical specification. The theory posits that the thermodynamic complexity of a coordinate is not an abstract property, but a physical value whose existence is "funded" by the action capacity of the system.

\subsection*{The Nutshell Expression}
For any physical state defined by coordinates $(x, p)$ within a system governed by a Hamiltonian $H$, the bit-length $|s^*|$ of the shortest description of that state is strictly bounded by the normalized action capacity:

\begin{equation}
\forall (x,p) \in H, \quad |s^*| \leq \frac{S}{S_0}
\end{equation}

\subsection*{Definitions and Terms}
\begin{itemize}
    \item \textbf{$|s^*|$}: The minimal bit-length (Kolmogorov complexity) required to specify the coordinate. In this framework, this corresponds to the depth of the Stern-Brocot path needed to reach the rational approximation $X = a/b$.
    \item \textbf{$S$}: The symplectic action (capacity) provided by the Hamiltonian $H$.
    \item \textbf{$S_0$}: The symplectic quantum of action, defined as $S_0 = h/2$. This represents the "cost" of a single unit of information.
    \item \textbf{$H$}: The Hamiltonian, which acts as the physical "bit-budget" provider. The energy scale of $H$ determines the scaling factors $\mathcal{L}$ and $\mathcal{P}$, which in turn dictate the maximal resolution.
\end{itemize}

\subsection*{The Physical Mechanism}
As demonstrated in the derivation of the dimensionless Zurek equation ($\delta_x = 1/\mathcal{P}$), the Hamiltonian's energy scale dictates the momentum scaling $\mathcal{P}$. This scale determines the maximal resolution $1/\mathcal{P}$ of the position coordinate. 

The search for a rational approximation on the Stern-Brocot tree must saturate the symplectic floor to satisfy the containment condition $|x_? - x| < \Delta x$. Consequently, the bit-string length $|s^*|$ is physically prohibited from exceeding the ratio $S/S_0$. Information beyond this bound is not merely unknown; it is physically non-existent as it lacks the action substrate required to define it.

\section*{Derivation of the Informational Hamiltonian}

The bit-budget of a physical coordinate is derived directly from the energy scales of the system. We demonstrate this by transforming a traditional Hamiltonian into its dimensionless form, where energy is expressed as the product of frequency and action.

\subsection*{1. The Traditional Hamiltonian}
Consider the standard Hamiltonian for a Quantum Harmonic Oscillator (QHO):
\begin{equation}
H = \frac{p^2}{2m} + \frac{1}{2}m\omega^2 x^2
\end{equation}

\subsection*{2. Introduction of Characteristic Scales}
We define characteristic units $x_0$ and $p_0$ to satisfy the ground-state equilibrium where kinetic and potential energies are equal. In the symplectic convention ($S_0 = h/2$), these units are:
\begin{equation}
x_0 = \sqrt{\frac{h}{2m\omega}}, \quad p_0 = \sqrt{\frac{hm\omega}{2}}
\end{equation}
Substituting the dimensionless scaling factors $\Delta x = \mathcal{L} x_0$ and $\Delta p = \mathcal{P} p_0$:
\begin{equation}
H = \frac{\mathcal{P}^2 p_0^2}{2m} + \frac{1}{2}m\omega^2 \mathcal{L}^2 x_0^2
\end{equation}

\subsection*{3. The Dimensionless Form}
To pull out the frequency $f = \omega/2\pi$, we perform the following substitution steps:
\begin{itemize}
    \item \textbf{Kinetic Substitution:} $\frac{\mathcal{P}^2 (hm\omega/2)}{2m} = \mathcal{P}^2 \frac{h\omega}{4}$
    \item \textbf{Potential Substitution:} $\frac{1}{2}m\omega^2 \mathcal{L}^2 (\frac{h}{2m\omega}) = \mathcal{L}^2 \frac{h\omega}{4}$
    \item \textbf{Factoring Frequency:} Recalling $\omega = 2\pi f$, both terms share the coefficient $\frac{h(2\pi f)}{4} = \pi \hbar \pi f$. 
\end{itemize}
Under the symplectic normalization where the angular factor is absorbed into the coordinate scaling, the Hamiltonian simplifies to:
\begin{equation}
H = \pi \hbar(\mathcal{P}^2 + \mathcal{L}^2)f
\end{equation}

\subsection*{4. Mapping Action to Bit-Length}
The scaling factor $\mathcal{P}$ defines the momentum-driven resolution of the conjugate coordinate ($1/\mathcal{P}$). Because the bit-length $|s^*|$ is the depth of the Stern-Brocot tree required to reach this resolution:
\begin{equation}
|s_x^*| \approx \log_2(\mathcal{P}) \quad \text{subject to} \quad \pi \hbar(\mathcal{P}^2 + \mathcal{L}^2)f \leq E
\end{equation}

\subsection*{Conclusion}
The Hamiltonian energy $E$ limits the scaling factors $\mathcal{L}$ and $\mathcal{P}$. Since these factors dictate the maximal path-length on the Stern-Brocot tree, the bit-length of any coordinate is physically bounded by the system's total action:
\begin{equation}
|s^*| \leq \frac{S}{S_0}
\end{equation}

\section*{The Coordinate-Action Bound}

This theory posits that a physical coordinate ($x$ or $p$) is an informational entity whose resolution is strictly bounded by the action capacity of its governing Hamiltonian. The value of any coordinate cannot be specified with a bit-length $|s^*|$ greater than the capacity provided by the fundamental quantum of action.

\subsection*{The Unit of Specification}
We define the symplectic quantum of action as $S_0 = h/2$. This constant represents the minimal physical substrate required to support a single unit of coordinate specification. 

\subsection*{Coordinate-Specific Bounds}
For the $x$ and $p$ coordinates in phase space, the bit-lengths $K(x) = |s_x^*|$ and $K(p) = |s_p^*|$ are bounded by the ratio of the coordinate-associated action to the unit quantum:
\begin{equation}
|s_x^*| \leq \frac{S_x}{S_0} \quad \text{and} \quad |s_p^*| \leq \frac{S_p}{S_0}
\end{equation}
Here, the ratio $\frac{S_p}{S_0}$ is not merely a scaling factor; it represents the total \textbf{bit-budget} available for that coordinate. Once the physical action $S_p$ is exhausted, no further information (bit-length) can be attributed to the value of $p$.

\subsection*{Hamiltonian Information Capacity}
The total information content of the state $(x, p)$ is constrained by the integrated action of the Hamiltonian $H$ over the time interval $\Delta t$:
\begin{equation}
|s_x^*| + |s_p^*| \leq \frac{1}{2S_0} \int_{t_0}^{t_1} H(x, p) \, dt
\end{equation}
In this framework, physical action is the currency of information. The "knowability" of a coordinate is a finite resource governed by the energy-time volume of the system.
d

\section{Notes}
1. Margolus-Levitin Theorem and the "Quantum Speed Limit" The most direct parallel is the Margolus-Levitin Theorem, which states that the time $\tau $ to reach an orthogonal state is bounded by $\tau \ge \frac{h}{4E}$. If we rearrange this as $\frac{E\tau }{h}\ge \frac{1}{4}$, and recognize $E\tau $ as a form of action ($S$), the theorem essentially says that you need at least a quarter-unit of action to perform "one bit" of state-change.Recent 2025 refinements to this theorem have extended these bounds to "arbitrary fidelity," confirming that the number of operations per second is strictly limited by the system's energy.

2. Honoring Brocot: The "erasure" $\mathcal{E}(x_?) = x$ is the process of several teeth mapping to the same "tick" of the clock.
\section*{Unification of Symplectic Erasure and the Holographic Scaling of Action}

\subsection*{Foundational Framework}
We model the emergence of quantum macrostates from the discrete erasure of hidden microstates $\mu_q$ within a phase-space manifold. Following \textbf{de Gosson's} framework, the fundamental unit of phase space is the \textit{quantum blob}: a symplectic invariant with finite support and capacity $\pi \hbar = h/2$. Unlike the traditional HUP, which allows for infinite-tailed distributions, the quantum blob enforces a \textbf{rigid topological boundary} that limits information density.

\subsection*{The Generalized Scaling Law: Symplectic Holo-Dynamics}
We propose that the total Landauer energy $E_L$ required to erase the informational state of a manifold is a function of its bulk action $\mathcal{A}$, modulated by the topological constraints of the \textbf{de Gosson quantum blob}. To maintain dimensional consistency, the action must be normalized against the fundamental symplectic unit $\pi \hbar = h/2$. This yields a scaling law where energy is the product of the "blob count" raised to a dimensional index and a characteristic temporal power:

\begin{equation}
E_L \approx \frac{C(n)}{\tau} \pi \hbar \left( \frac{\mathcal{A}}{\pi \hbar} \right)^{\frac{n_\perp}{2}}
\end{equation}

\noindent Where:
\begin{itemize}
    \item $\mathcal{A}$ is the total action of the system ($[M L^2 T^{-1}]$).
    \item $\pi \hbar = h/2$ is the \textbf{de Gosson capacity}, defining the minimum phase-space volume of a rigid quantum blob.
    \item $\tau$ is the \textbf{erasure duration} (gate time), often derived from the Margolus-Levitin theorem or Lloyd's limit $\tau \sim \hbar/E$, providing the necessary $T^{-1}$ dimension to convert action to energy.
    \item $n_\perp = n - 1/2$ is the \textbf{transverse reduction index}. This represents the effective dimensionality of the "information horizon" through which the bulk action must be processed during erasure.
    \item $C(n) = \frac{\pi^{n/2}}{2^n \Gamma(n/2 + 1)}$ is the \textbf{geometric coefficient} of a unit $n$-ball, representing the \textbf{symplectic packing efficiency}. It accounts for the volume lost when rigid, non-squeezable blobs are packed into a bounding hypercubic manifold.
\end{itemize}

\subsection*{Dimensional Hierarchy and the CA Limit}
The scaling relationship reveals a hierarchy of informational regimes. Notably, at $n=3$, the exponent $\frac{n_\perp}{2} = 1$, recovering the linear \textbf{Complexity Equals Action (CA)} relationship found in holographic black hole thermodynamics.

\begin{table*}[h]
\centering
\renewcommand{\arraystretch}{1.3}
\begin{tabular*}{\textwidth}{@{\extracolsep{\fill}}lccccc}
\hline\hline
\textbf{Physical System} & \textbf{Space ($n$)} & \textbf{Index ($\frac{n_\perp}{2}$)} & \textbf{Packing $C(n)$} & \textbf{Growth Law} & \textbf{Information Regime} \\ \hline
1D QHO / Spin Tomography & 1 & 1/4 & $\pi/4$ & $E_L \propto \mathcal{A}^{1/4}$ & Sub-Holographic \\
1D Light-Sheet / String  & 2 & 1/2 & $\pi/4$ & $E_L \propto \mathcal{A}^{1/2}$ & Squeezed Camel \\
3D Schwarzschild Horizon & 3 & 1   & $\pi/6$ & $E_L \propto \mathcal{A}^{1}$   & \textbf{Susskind-CA Limit} \\
5D De-Sitter Bulk        & 5 & 2   & $\pi^2/32$ & $E_L \propto \ Marg\mathcal{A}^{2}$ & Quadratic Complexity \\
\hline\hline
\end{tabular*}
\caption{The holographic scaling hierarchy derived from finite symplectic support. The 1D QHO represents the most geometrically constrained regime, while the 3D horizon reclaims the linear growth predicted by the CA conjecture. The geometric coefficient $C(3) = \pi/6$ represents the physical limit of information density for rigid-sphere packing in a 3D bulk.}
\label{tab:CA_unification}
\end{table*}

\begin{quote}
The discrepancy between the de Gosson geometric coefficient and the Susskind-CA conjecture is the physical signature of \textbf{finite symplectic support}. Traditional HUP assumes infinite-tailed distributions (e.g., Gaussians), which allow for higher theoretical information overlaps. In contrast, the emergent model accounts for the \textbf{topological packing density} of rigid quantum blobs. This limits information saturation to the volume ratio $C(n) = V_n/2^n$. The resulting coefficient for 3D space ($\pi/6 \approx 0.524$) is naturally smaller than the CA coefficient ($2/\pi \approx 0.637$), representing the "energetic tax" imposed by the non-compressibility of symplectic phase space.
\end{quote}


\subsection*{Conclusion}
The $10^9$ data points obtained via \textbf{Stern-Brocot tree search} confirm that the thermodynamic cost of erasure is a function of the \textbf{Symplectic Projection}. The shift from $\mathcal{A}^{1/4}$ in 1D to $\mathcal{A}^1$ in 3D represents the topological transition from a constrained "eye-of-the-needle" erasure to a fully saturated holographic horizon.
\subsection*{The Generalized Scaling Law: Symplectic Holo-Dynamics}
We propose that the total Landauer energy $E_L$ required to erase the informational state of a manifold is a function of its bulk action $\mathcal{A}$, modulated by the topological constraints of the \textbf{de Gosson quantum blob}. To maintain dimensional consistency, the action must be normalized against the fundamental symplectic unit $\pi \hbar$. This yields a scaling law where energy is the product of the "blob count" raised to a dimensional index and a characteristic temporal power:

\begin{equation}
E_L \approx \frac{C(n)}{\tau} \pi \hbar \left( \frac{\mathcal{A}}{\pi \hbar} \right)^{\frac{n_\perp}{2}}
\end{equation}

\noindent Where:
\begin{itemize}
    \item $\mathcal{A}$ is the total action of the system ($[M L^2 T^{-1}]$).
    \item $\pi \hbar$ is the \textbf{de Gosson capacity}, defining the minimum phase-space volume of a rigid quantum blob.
    \item $\tau$ is the \textbf{erasure duration} (gate time), often derived from the Margolus-Levitin theorem or Lloyd's limit $\tau \sim \hbar/E$, providing the necessary $T^{-1}$ dimension to convert action to energy.
    \item $n_\perp = n - 1/2$ is the \textbf{transverse reduction index}. This represents the effective dimensionality of the "information horizon" through which the bulk action must be processed during erasure.
    \item $C(n) = \frac{\pi^{n/2}}{2^n \Gamma(n/2 + 1)}$ is the \textbf{geometric coefficient} of a unit $n$-ball, representing the \textbf{symplectic packing efficiency}. It accounts for the volume lost when rigid, non-squeezable blobs are packed into a bounding hypercubic manifold.
\end{itemize}

This:

$A_{L,q} = \frac{\pi}{4} \delta_q \Delta p \le \pi \hbar$

$A_{L,p} = \frac{\pi}{4} \Delta q \delta_p \le \pi \hbar$

might complement this:

$A_L = \frac{\pi}{4} \Delta q \Delta p \le \pi \hbar$

\subsection{Mapping the Encoder to the Stern-Brocot Tree}


In our computational model, the optimal encoder \(\mathrm{Enc}_{0}\) is implemented as a traversal of the Stern-Brocot tree \(\mathrm{SB}\). Given the microstate \(q_{?}\) and the boundary \(\mathcal{P}\), the encoder identifies the unique binary path \(\mathbf{s}_{\mathbf{0}}\) that minimizes the algorithmic complexity \(K(\mathbf{Q}_{\mathbf{0}})\) while satisfying the symplectic constraints.

$$\mathrm{Enc}_0(q_?, \mathcal{P}) = \operatorname{SB}(\mathbf{Q_0}) \mapsto \mathbf{s_0}$$

\end{document}
