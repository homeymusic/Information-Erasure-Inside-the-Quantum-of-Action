\documentclass[%
 aps, reprint, prl, amsmath,amssymb
]{revtex4-2}

\usepackage{graphicx}% Include figure files
\graphicspath{{figures/}} 
\usepackage{dcolumn}% Align table columns on decimal point
\usepackage{bm}% bold math
\usepackage{orcidlink}
\usepackage{mathtools}

% -------------- Watermark ----------------
\usepackage[style=iso]{datetime2}
\usepackage{FiraMono} 
\usepackage{tikz}
\usepackage{transparent} 

% Use shipout/background to ensure it stays behind text/figures
\AddToHook{shipout/background}{
  \begin{tikzpicture}[remember picture, overlay]
    \node [
      rotate=55,
      scale=1.2,
      text opacity=0.15,      
      color=gray!50,          
      font=\fontfamily{FiraMono-TLF}\selectfont\bfseries,
      align=center
    ] at (current page.center) {
        {\fontsize{60}{70}\selectfont WORKING DRAFT} \\ [0.8cm]
        {\Huge \DTMnow} \\
        {\Huge doi.org/10.17605/OSF.IO/EV8H6} \\ [0.8cm]
        {\Large Computational Complex Systems Laboratory} \\
        {\Large Homey Music, Detroit, USA} \\ [0.2cm]
         {\transparent{0.25}\includegraphics[width=1.0cm]{logo.png}}
    };
  \end{tikzpicture}
}
% -----------------------------------------


\DeclareMathOperator*{\argmin}{argmin}
\begin{document}
\title{Information Erasure Inside the Quantum of Action}

\begin{abstract}
By maximizing entropy through Landauer information erasure inside de Gosson's quantum of action—where states are operationally indistinguishable—we show that the continuum collapses into discrete conjugate observables. Consistent with the experimental difficulty in isolating pure states, numerical results confirm that as action increases, Schrödinger-eigenstates become vanishingly sparse configurations—not privileged solutions—among the expanding possibilities in phase space. At the entropic limit of action, the proposed model provides a thermodynamic-geometric correspondence to standard quantum mechanics.
\end{abstract}

\author{Brian S. Mulloy\orcidlink{0000-0002-1803-3172}}
\email{brian@homeymusic.com}
\affiliation{Computational Complex Systems Laboratory\\Homey Music, Corktown, Detroit, MI, USA}
\date{\today}

\maketitle

\section{Foundational Works}
\begin{itemize}
    \item Landauer \cite{Landauer1961}
    \item de Gosson \cite{DeGosson2003}
    \item Heisenberg \cite{Heisenberg1927}
    \item Stern \cite{Stern1858}
    \item Brocot \cite{Brocot1861}
    \item Stolzenburg \cite{Stolzenburg2015}
    \item Plenio and Vitelli \cite{PlenioVitelli2001}
\end{itemize}


\bibliography{main} 
\end{document}
