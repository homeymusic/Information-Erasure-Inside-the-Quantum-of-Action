\documentclass[%
 aps, reprint, prl, amsmath,amssymb
]{revtex4-2}

\usepackage{graphicx}% Include figure files
\usepackage{dcolumn}% Align table columns on decimal point
\usepackage{bm}% bold math
\usepackage{orcidlink}
\usepackage{mathtools}
\graphicspath{{figures/}}
\DeclareMathOperator*{\argmin}{argmin}
\begin{document}

\title{Quantized Phase-Space from Information Erasure}

\begin{abstract}
The Second Law of Thermodynamics drives systems toward maximum entropy. We show that, when applied to the quantum of action in symplectic phase space, this imperative maximizes total entropy by erasing the surplus information that is operationally indistinguishable within minimal Heisenberg cells thereby collapsing the continuum into a quantized geometry. Through numerical analysis of the harmonic oscillator, we demonstrate that this emergent geometry recovers the marginals of the Wigner quasiprobability distribution. These results suggest that quantized phase space emerges from thermodynamic necessity, providing quantum mechanics with a geometric correspondence that operates independently from the postulates of probability waves and superposition.
\end{abstract}

\author{Brian S. Mulloy\orcidlink{0000-0002-1803-3172}}
\email{brian@homeymusic.com}
\affiliation{Computational Complex Systems Laboratory\\Homey Music, Corktown, Detroit, MI, USA}
\date{\today}

\maketitle

\begin{figure}[htbp]
    \centering
    \includegraphics[width=\linewidth]{quantum_of_action.pdf}
    \caption{Phase-space representation of the symplectic quantum of action $\mathcal{A}=h/2$. The shaded region represents the "quantum blob" with area constraint $x_0 p_0 = h/2$.}
    \label{fig:quantum_blob}
\end{figure}

\section{Notation}

We define $x_{?}$ as an unknowable value, a pre-erasure microstate: the original, fine-grained coordinate of the system. We say that information has been erased during an observation because we will never find out what the state $x_{?}$ was originally. Inside the quantum of action $\mathcal{A} = x_0 p_0 = h/2$ (the coarse-grained minimal Heisenberg cell), $x_{?}$ is operationally indistinguishable from the resulting observed state $x$ such that 
$$|x_{?}-x|\Delta p<h/2.$$ 

Because the specific value of $x_{?}$ is rendered forever unknowable by this many-to-one compression, the erasure is logically irreversible. 

 The difference between $x$ and $x_{?}$ is that the algorithmic complexity $|b_?|$ of $x_?$ is greater than the algorithmic complexity of the minimal algorithmic complexity value $|b_?| > |b^*|$ within the quantum of action.

\medskip

We define $b$ as the program bit string of length $|b|$ required to specify 
the observed state $x$. Because every additional bit in the string incurs 
a thermodynamic erasure cost of $kT \ln 2$, the physical system is subject 
to an algorithmic-thermodynamic pressure. It is therefore a thermodynamic 
necessity that from the ensemble of available microstates 
$\{x_? : |x_? - x| \Delta p < h/2 \}$, the observed value $x$ must 
correspond to the unique bit string $\mathbf{b^*}$ of minimal length 
$|\mathbf{b^*}|$. This is expressed as the algorithmic mapping 
$$x = \mathcal{U}(\mathbf{b^*})$$ 
where $\mathcal{U}$ represents the universal physical constructor that 
identifies the state $x$ as the output of the minimal program $\mathbf{b^*}$ 
that satisfies the resolution limit $|x_{?} - x| \Delta p < h/2$. In this 
framework, the transition to a quantized state is a thermodynamic selection of the 
lowest-complexity representation permitted by the system's energy scale. 
This selection is governed by the principle that the universe naturally 
settles into the state of \textit{minimal informational action} [1]. As 
Landauer established, longer strings represent higher-potential energy states 
for the coupled system-reservoir ensemble, and thus $\mathbf{b^*}$ is the 
only state whose entropy can be fully dissipated by the local thermal 
background [2]. Furthermore, in the framework of the 
\textit{Foundations of Algorithmic Thermodynamics} [3], $\mathbf{b^*}$ 
represents the "coarsest" algorithmic descriptor that remains invariant 
under environmental interaction. Any higher-complexity representation $b_{n}$ 
is thermodynamically unstable; the extra information is effectively 
"scrambled" by the thermal reservoir, forcing the system to halt at the 
minimal bit string $\mathbf{b^*}$. It is this thermodynamic necessity of 
erasing the original state and selecting the shortest description 
$\mathbf{b^*}$ that forces the phase-space to emerge as a quantized 
structure rather than a continuous one.

\medskip

The implementation of the universal constructor $\mathcal{U}$ is operationally realized through the Stern-Brocot decoding function $\operatorname{SB}$, which maps the minimal program $\mathbf{b^*}$ to a unique rational $X \in \mathbb{Q}$. The constructor is thus defined by scaling this rational address with the fundamental length unit:
\begin{equation}
\mathcal{U}(\mathbf{b^*}) = x_0 \operatorname{SB}(\mathbf{b^*})
\end{equation}
In this framework, the Stern-Brocot tree—a complete binary search tree of all positive rationals—acts as the decoding architecture for phase-space. The minimal program $\mathbf{b^*}$ is an ordered bit string where each bit represents a discrete branching instruction: a '0' for a left (L) traversal and a '1' for a right (R) traversal. The constructor $\mathcal{U}$ reads these instructions to navigate from the root toward the rational $X$ that identifies the quantized state.
\begin{equation}
x = x_0 \operatorname{SB}(\mathbf{b^*}) = x_0 X
\end{equation}
This traversal is not indefinite; it is governed by the thermodynamic necessity to halt. As the tree is descended, the bit-length $|\mathbf{b^*}|$ increases, incurring an cumulative erasure cost of $|\mathbf{b^*}|kT\ln 2$. Consequently, the algorithmic thermodynamics select the exact level where the resolution limit $|x_{?} - x_0 \operatorname{SB}(\mathbf{b^*})| \Delta p < h/2$ is first satisfied. By embedding the Stern-Brocot tree directly into the constructor, the physical system ensures that the resulting quantized value $x$ is the most algorithmically efficient representation of the  available microstates 
$\{x_? : |x_? - x_0 X| \Delta p < h/2 \}$. This mechanism minimizes the complexity of the physical record by selecting the most accessible rational convergent within the resolution bound, effectively forcing the phase-space to emerge as a quantized structure from the thermodynamic limit of information erasure.

\medskip
The physical impossibility of resolving irrational states is best illustrated by the Golden Ratio, 
$$\phi \approx 1.618 \ 033 \ 988 \ 749 \ \dots$$ 
In the Stern-Brocot implementation of $\mathcal{U}$, the program required to compute $x = \phi x_0$ is a non-terminating instruction set $b_{\phi} = (1, 0, 1, 0, \dots)$. This infinite string exposes a critical divergence between the Shannon entropy $H_{Sh}$---the \textit{ensemble average} of information---and the algorithmic entropy $H_{alg}$, which quantifies the physical complexity of the \textit{individual microstate}. 

Following the framework of algorithmic thermodynamics \cite{Ebtekar2025}, we define the relationship between the ensemble average and the individual microstate complexity:
\begin{equation}
    H_{Sh} = -\sum_{i \in \{0,1\}} w_i \log_2 w_i, \quad H_{alg}(x_{?}) = |b|
\end{equation}
where $w_i$ represents the statistical probability of a branching instruction $i$ within the ensemble, and $H_{alg}(x_{?})$ is the actual length of the instruction set $b$ required to manifest the specific microstate $x_?$. While the Shannon entropy of the alternating rule for $\phi$ remains finite and low, the physical Landauer cost of manifesting the individual microstate is driven by the absolute bit-length of its program:
\begin{equation}
    \mathcal{W} = \lim_{|b| \to \infty} |b| k T \ln 2 = \infty
\end{equation}

This distinction is fundamental: whereas Shannon entropy measures our mean ignorance of an ensemble, the algorithmic complexity $|b|$ quantifies the actual thermodynamic weight of the specific microstate at each step of its resolution. As Zurek established in his formulation of \textit{physical entropy} \cite{Zurek1989}, the total entropy $S_d = H_{alg} + H_{Sh}$ must account for the information already manifested in the record. Because every additional bit in $b$ incurs a cumulative energy cost, the physical system is subject to an informational pressure that favors the state of minimal complexity. Since no local thermal reservoir possesses the infinite energy capacity required to finalize the non-terminating program $b_\phi$, the constructor $\mathcal{U}$ is forced to \textit{halt} at the minimal program $\mathbf{b^*}$---the unique rational convergent that satisfies the \textbf{Heisenberg resolution limit} $|x_{?} - x| \Delta p < h/2$ with the least thermodynamic work. This thermodynamic barrier proves that irrational coordinates are physically unrealizable, leaving the quantized rational record as the only possible state of the phase-space.


\subsection{Statistical Heisenberg Uncertainty Principle}

$$
\sigma_x \sigma_p \ge \frac{\hbar}{2}
$$

\subsection{Symplectic Heisenberg Uncertainty Principle}

\subsubsection{The Quantum of Action}

$$
\mathfrak{a} = x_0 p_0 = \frac{h}{2}
$$

\subsubsection{The Symplectic Capacity}

$$
\mathcal{A} = \mathfrak{a} ( \mathcal{L} \times \mathcal{P}) \ge \frac{h}{2}
$$

$$
\mathcal{A} = c(\Omega)
$$

where $L$ and $P$ are dimensionless scaling factors of the characteristic position $x_0$ and momentum $p_0$.

$$
\Delta x \Delta p \ge \frac{h}{2}
$$

$$
\Delta x := \mathcal{L} x_0 \quad \Delta p := \mathcal{P} p_0
$$

\subsection{Quantum Harmonic Oscillator}

$$
H = \frac{\mathcal{P}^2 p_0^2}{2m} + \frac{1}{2}m\omega^2 \mathcal{L}^2 x_0^2
$$

$$
H = \frac{h}{2}(\mathcal{P}^2 + \mathcal{L}^2)f
$$

\subsection{Fundamental Fraction}

\begin{quote}

We say that information has been erased during the compression because we will never find out where the molecule was originally.

\hfill --- \textit{The Physics of Forgetting}

\end{quote}

We say that information has been erased $|b|kT\ln{2}$ during the observation $x_? \rightarrow x = X x_0, \ X \in \mathbb{Q}$ because we will never find out what the state $x_?$ was originally. State values within the coarse-grained Heisenberg uncertainty cell (quantum blob) are operationally indistinguishable 
$|x_? - x| \Delta p < h/2$. Each $X$ corresponds to a bit string $b$ which is unique for the given physical system. For any physical system $\Delta x := \mathcal{L} x_0 \ \Delta p := \mathcal{P} p_0$ is defined by the Hamiltonian. The dimensionless scaling factor $\mathcal{P}$ indicates the momentum and also the resolution limit of the conjugate observable and is determined by the Hamiltonian's energy scale. We find $X$ and $b$ using the Stern-Brocot tree such that $\delta_x = |x_?/x_0 - X| < 1 / \mathcal{P} \implies \delta_x \Delta p < h/2$, using the hard symplectic boundary $h/2$ of the Heisenberg cell as opposed to the fuzzy statistical one $\hbar/2$. The spread of the position and momentum as $\mathcal{P}$ increases the ensemble Shannon entropy while the increasing resolution $1/\mathcal{P}$ increases the algorithmic entropy required to erase the information about the unknown state.

$$
|x_? - x| < \Delta x
\implies
|x_? - x| \Delta p < \frac{h}{2}
$$


$$
\delta_x = 
\frac{|x_? - x|}{x_0} = 
\frac{\left | x_? - \frac{a}{b} x_0 \right |}{x_0} = 
x_0 \frac{\left | \frac{x_?}{x_0} - \frac{a}{b} \right | }{x_0} =
\left | \frac{x_?}{x_0} - \frac{a}{b} \right |
$$

$$
\left | \frac{x_?}{x_0} - \frac{a}{b} \right | x_0 p_0 P < \mathcal{A}
$$

To satisfy the containment condition $|x_?-x|<\Delta x$ with minimal algorithmic overhead, the search for a rational approximation must saturate the symplectic floor. At this limit of information erasure, the coarse-grained action converges to the quantum of action: $\mathcal{A}\rightarrow \mathfrak{a}=h/2$.

$$
\left | \frac{x_?}{x_0} - \frac{a}{b} \right | \mathfrak{a} P < \mathfrak{a}
$$


$$
\left | \frac{x_?}{x_0} - \frac{a}{b} \right | \frac{h}{2} P < \frac{h}{2}
$$

$$
\delta_x = \left | \frac{x_?}{x_0} - \frac{a}{b} \right | < \frac{1}{P}
$$

We have a dimensionless form of Zurek's equation $\delta_x = 1/P$.

\subsection{Symplectic Symmetry}
The derivation for the conjugate variable $p$ follows by the same logic:
$$ \delta_p = \frac{|p_? - p|}{p_0} = \left | \frac{p_?}{p_0} - \frac{c}{d} \right | $$
Applying the same boundary condition for the blob area:
$$ \delta_p p_0 \Delta x \ge \frac{h}{2} \implies \left | \frac{p}{p_0} - \frac{c}{d} \right | x_0 p_0 L \ge \frac{h}{2} $$
Substituting the fundamental action unit $x_0 p_0 = h/2$ yields the symmetric result:
$$ \delta_p = \left | \frac{p_?}{p_0} - \frac{c}{d} \right | \ge \frac{1}{L} $$
We have a dimensionless form of Zurek's equation $\delta_p = 1/L$.
This demonstrates that the resolution of phase space is constrained by the same algorithmic entropy in both coordinates.

\subsection{On Sub-Planck Scales}
We suggest that the sub-Planck structures identified by Zurek are not violations of the fundamental quantum of action, but rather a manifestation of symplectic squeezing. By allowing the quantum blob $a=h/2$ to evolve into an elliptical geometry defined by our scaling factors $L$ and $P$, one can recover Zurek’s high-resolution interference features. In this view, the sub-Planck scale actually represents a one-dimensional resolution limit $\delta_{x}$ or $\delta_{p}$ of a squeezed symplectic invariant, rather than a reduction of the total phase-space area below the de Gosson boundary.


\end{document}
