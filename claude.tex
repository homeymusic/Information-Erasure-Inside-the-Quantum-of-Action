\documentclass[%
 aps, reprint, prl, amsmath,amssymb
]{revtex4-2}

\usepackage{graphicx}
\usepackage{dcolumn}
\usepackage{bm}
\usepackage{orcidlink}
\usepackage{svg}
\usepackage{mathtools}
\usepackage{amsmath}
\DeclareMathOperator*{\argmin}{argmin}
\begin{document}

\title{Landauer-Heisenberg Constraint: Elementary Quantum Structure from Fundamental Fractions}

\author{Brian McAuliff Mulloy\orcidlink{0000-0002-1803-3172}}
\email{bmulloy@umich.edu}
\affiliation{Detroit, MI, USA}
\date{\today}

\begin{abstract}
Phase space quantization emerges as a thermodynamic necessity when Landauer's principle is applied to state description within minimal Heisenberg cells.
\end{abstract}

\maketitle

The Heisenberg uncertainty principle $\Delta x \Delta p \ge \hbar/2$ sets the resolution limit of phase space. Within the saturated minimal cell $\Delta x \Delta p = \hbar / 2$, all states are operationally indistinguishable [Fig.~\ref{fig:LandauerHeisenbergCell}]. Yet the principle remains silent on which state the system actually occupies.

Any refinement beyond the uncertainty bound corresponds to inaccessible information. When coupled to an environment, this surplus information is necessarily erased through irreversible coarse graining. By Landauer's principle ($E \ge k_B T \ln 2$ per bit erased), this erasure costs energy proportional to description length. The physically realized state must therefore minimize its informational burden.

We identify states with \textit{fundamental fractions} $Q_0 = a/b \in \mathbb{Q}_0 \subset \mathbb{Q}$: the minimal description length rationals in the Stern-Brocot tree. Each branching step adds one bit; the description length $L(Q_{0})$ equals tree depth, giving Landauer cost $E_0 = L(Q_0) k_B T \ln 2$.

\begin{figure}[tbp]
    \centering
    \includesvg[width=0.62\linewidth]{LandauerHeisenbergCell}
    \caption{\textbf{The Landauer-Heisenberg Phase-Space Cell.} Within $\Delta x \Delta p = \hbar/2$, the physical state (solid dot) is the minimal description length rational $a/b$. High-precision surplus information (faint point) is thermodynamically erased to this fundamental fraction.}
    \label{fig:LandauerHeisenbergCell}
\end{figure}

The selection rule follows from operational indistinguishability. Define the approximation error $\delta \tilde{x} = x_0 | x/x_0 - a/b |$ and momentum uncertainty $\Delta p = n_{p_0} \hbar/x_0$, where $x_0$ and $p_0 = \hbar/x_0$ are characteristic scales. For the state to remain within the Heisenberg cell:
\begin{equation}
\label{eq:HeisenbergBox}
\left| \frac{x}{x_0} - \frac{a}{b} \right| < \frac{1}{2 n_{p_0}} \implies \delta \tilde{x} \Delta p < \frac{\hbar}{2}.
\end{equation}

This constraint, satisfied by the minimal path rational in the Stern-Brocot tree, identifies the fundamental fraction that minimizes thermodynamic cost while respecting the quantum of action. Quantization is not empirical observation but thermodynamic necessity: the quantum of action and the cost of information jointly bound phase space to a discrete set of fundamental fractions.

Using $10^{8}$ computational data points, we verify this mapping recovers the discrete structures of elementary quantum systems.

\textbf{[Forthcoming: Numerical verification to be inserted here.]}

\nocite{*}
\bibliography{apssamp}

\end{document}