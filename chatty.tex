\documentclass[%
aps, reprint, prl, amsmath,amssymb
]{revtex4-2}

\usepackage{graphicx}
\usepackage{bm}
\usepackage{orcidlink}
\usepackage{mathtools}

\begin{document}

\title{Landauer--Heisenberg Constraint: Quantum States from Minimal Description Length}

\author{Brian McAuliff Mulloy\orcidlink{0000-0002-1803-3172}}
\email{[bmulloy@umich.edu](mailto:bmulloy@umich.edu)}
\affiliation{Detroit, MI, USA}
\date{\today}

\begin{abstract}
We show that quantization within a minimal Heisenberg phase-space cell follows from Landauer’s principle: among operationally indistinguishable states, thermodynamic stability selects the representation of minimal description length.
\end{abstract}

\maketitle

The Heisenberg uncertainty principle bounds phase-space resolution by
\begin{equation}
\Delta x \Delta p \ge \hbar/2 .
\end{equation}
When saturated, all states within the minimal cell are operationally indistinguishable. Quantum mechanics fixes the cell’s area but does not specify the informational identity of the state it contains.

Landauer’s principle supplies the missing constraint. Any logically irreversible operation, including the erasure of information, dissipates energy
\begin{equation}
E \ge k_B T \ln 2 \quad \text{per bit}.
\end{equation}
Information that cannot be operationally distinguished cannot be stabilized against environmental coupling and is therefore necessarily erased. Within a minimal Heisenberg cell, any state description exceeding the resolution bound constitutes surplus information and incurs an unavoidable thermodynamic cost.

This establishes a variational principle: \emph{among all representations compatible with} $\Delta x \Delta p = \hbar/2$, the physically realized state minimizes description length. Higher-complexity encodings correspond to hidden degrees of freedom that must be continually protected from thermalization, rendering them thermodynamically unstable.

We identify the physical state with a \emph{fundamental fraction} $Q_0 = a/b$, a rational representative of minimal description length within the cell. The rationals are ordered by the Stern--Brocot tree, where each branching corresponds to a single bit. The description length
\begin{equation}
L(Q_0) = \text{depth}(Q_0)
\end{equation}
measures the algorithmic cost of specifying the state.

Let $x_0$ and $p_0=\hbar/x_0$ denote characteristic scales. Defining
\begin{align}
\delta \tilde{x} &\equiv x_0 \left| \frac{x}{x_0} - \frac{a}{b} \right| ,\
\Delta p &\equiv n_{p_0} p_0 ,
\end{align}
operational indistinguishability requires
\begin{equation}
\label{eq:HeisenbergBox}
\left| \frac{x}{x_0} - \frac{a}{b} \right| < \frac{1}{2 n_{p_0}}
;;\Longrightarrow;;
\delta \tilde{x} \Delta p < \frac{\hbar}{2}.
\end{equation}
The selected $Q_0$ is the minimal-depth rational satisfying Eq.~\eqref{eq:HeisenbergBox}.

The associated Landauer cost is
\begin{equation}
E_0 = L(Q_0), k_B T \ln 2 .
\end{equation}
States requiring longer descriptions are exponentially suppressed by their thermodynamic penalty. Quantization thus emerges not as an imposed discretization, but as the joint consequence of finite phase-space resolution and the energetic cost of information.

We show numerically that this Landauer--Heisenberg constraint reproduces the discrete structures of elementary quantum systems.

\bibliography{apssamp}

\end{document}
